\documentclass[11pt]{article}
\usepackage{amssymb,latexsym,amsmath,amsthm,graphicx, cite}
 \author{Chris Camano: ccamano@sfsu.edu}
 \title{MATH 335  lecture 13 }
 \date

\usepackage{mathptmx}
\usepackage{multirow}
\usepackage{float}
\restylefloat{table}
\hoffset=0in
\voffset=-.3in
\oddsidemargin=0in
\evensidemargin=0in
\topmargin=0in
\textwidth=6.5in
\textheight=8.8in
\marginparwidth 0pt
\marginparsep 10pt
\headsep 10pt

\theoremstyle{definition}  % Heading is bold, text is roman
\newtheorem{theorem}{Theorem}
\newtheorem{corollary}{Corollary}
\newtheorem{defn}{Definition}
\newtheorem{example}{Example}
\newtheorem{proposition}{Proposition}



\newcommand{\Z}{\mathbb{Z}}
\newcommand{\N}{\mathbb{N}}
\newcommand{\Q}{\mathbb{Q}}
\newcommand{\R}{\mathbb{R}}
\newcommand{\C}{\mathbb{C}}

\newcommand{\lcm}{\mathrm{lcm}}
\newcommand{\bigline}{\\\noindent\makebox[\linewidth]{\rule{\paperwidth}{0.4pt}}\\}


\setlength{\parskip}{0cm}
%\renewcommand{\thesection}{\Alph{section}}
\renewcommand{\thesubsection}{\arabic{subsection}}
\renewcommand{\thesubsubsection}{\arabic{subsection}.\ar  abic{subsubsection}}
\bibliographystyle{amsplain}

%\input{../header}


\begin{document}
\maketitle
\bigline
\section{Sub Groups and their properties}
\bigline
\defn Subgroup\\
Let G be a group. A subset H of G is called a subgroup if H itself is a group, when we restrict the group operation to H.\\ \bigline
\textbf{Let G=$S_3$} the group operation of this group is function composition, thus to consider subgroups of this group we will first consider which subsets still are verifiable groups under function composition.
Consider:
\[
\left\{
\begin{bmatrix}
  1&2&3\\
  1&2&3
\end{bmatrix},
\begin{bmatrix}
  1&2&3\\
  2&3&1
\end{bmatrix},
\begin{bmatrix}
  1&2&3\\
  2&3&1
\end{bmatrix}^2=\begin{bmatrix}
  1&2&3\\
  3&1&2
\end{bmatrix}
\right\}
\]\\
This Group is a subgroup since for all elements we are closed under function composition\\
\bigline
For all subgroups the identity element must be an element of the subgroup Since we still are using the same binary operator.\\\\
The subgroup consisting of an element the identity and its inverse always forms a group since H will be closed under the group operation due to the limited selection criteria from the set. \\\\
Every group has at minimum two subgroups. The first is the set of just the identity, the trivial subgroup. The next subgroup is the entire group itself. (typically ignored)\\\\
If we have the case where an element's inverse is itself the subgroup consisting of that element and the identity forms a subgroup.
\bigline
Examples include cyclic group about 90 degree rotations in the complex unit circle in C and  the general linear group for 2x2 matrices. \\\\
\defn $SL_2$ The special linear group is a subgroup of the general linear group defined as follows.:
\[
  SL_2=\{A\in GL_2: det(A)=1\}
\]
Since we have:
\[
  \det(AB)=\det A\det B
\]
The group is closed under matrix multiplication and is concequently a subgroup of $SL_2$ \\
We also verify that taking the inverse of a matrix still places us in SL2 after the operation.This can be done by leveraging thefact that the coefficient over the inverse computtation begcomes 1 over 1 when det(A)=1 so we are still within $SL_2$
\\
\bigline
\proposition Let G be a group and let h be a subset of G then h is a subgroup of G if and only if:
\begin{enumerate}
  \item $e\in H$
  \item  $\forall\ h_1,h_2\in H \rightarrow h_1\circ h_2 \in H$
  \item if $h\in H$ then $h^{-1}\in H$
\end{enumerate}
\\
\bigline
\textbf{Example} Consider $M_2$ the set of all matrices under the binary operator of matrix addition\\
\begin{enumerate}
  \item The operator is closed since dimensions are preserved.
  \item The operation is associative by extension of the associativty of $\mathbb{R}$
  \item Identity element is the zero matrix
  \item the inverse element is -A\forall\ A \in M_2
\end{enumerate}
\proposition $GL_2\psubset M_2$ since $M_2$ is all 2x2 matrices. Is $GL_2$ a subgroup of $M_2$? Well, the zero matrix is not in $GL_2$ and concequently we do not have a subgroups
\\
\defn If H is a subgroup of G we write:
\[
  H\leq G
\]
\end{document}
