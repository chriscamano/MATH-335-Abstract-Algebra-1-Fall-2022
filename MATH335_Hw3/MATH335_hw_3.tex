\documentclass[11pt]{article}
\usepackage{amssymb,latexsym,amsmath,amsthm,graphicx, cite}
 \author{Chris Camano: ccamano@sfsu.edu}
 \title{MATH 335  Homework 3 }
 \date

\usepackage{mathptmx}
\usepackage{multirow}
\usepackage{float}
\restylefloat{table}
\hoffset=0in
\voffset=-.3in
\oddsidemargin=0in
\evensidemargin=0in
\topmargin=0in
\textwidth=6.5in
\textheight=8.8in
\marginparwidth 0pt
\marginparsep 10pt
\headsep 10pt

\theoremstyle{definition}  % Heading is bold, text is roman
\newtheorem{theorem}{Theorem}
\newtheorem{corollary}{Corollary}
\newtheorem{definition}{Definition}
\newtheorem{example}{Example}
\newtheorem{proposition}{Proposition}



\newcommand{\Z}{\mathbb{Z}}
\newcommand{\N}{\mathbb{N}}
\newcommand{\Q}{\mathbb{Q}}
\newcommand{\R}{\mathbb{R}}
\newcommand{\C}{\mathbb{C}}

\newcommand{\lcm}{\mathrm{lcm}}



\setlength{\parskip}{0cm}
%\renewcommand{\thesection}{\Alph{section}}
\renewcommand{\thesubsection}{\arabic{subsection}}
\renewcommand{\thesubsubsection}{\arabic{subsection}.\arabic{subsubsection}}
\bibliographystyle{amsplain}

%\input{../header}


\begin{document}

\maketitle
\begin{enumerate}

\item Let $A = \{a, b, c\}$ and $B = \{1,2,3,4\}$.
  \begin{itemize}
    \item[a)] Determine $A \times B$.\\\\
    $A\cross B=\{ (a,1),(a,2),(a,3),(a,4),(b,1),(b,2),(b,3),(b,4),(c,1),
    (b,2),(b,3),(b,4)\}$
    \item[b)] Draw $B \times \R$.\\\\\\\\\\\\\\

     \item[c)] Prove that if $C$ and $D$ are finite sets then $|C \times D| = |C| \cdot |D|$.
     \begin{proof}\\\\
       By the definition of the cartesian product for any fixed $c_o \in C$ there exist$ |D|$ corresponding ordered pairs of the form $(c_0,d), d\in D$ .\\\\ Since there are $|C|$ possible choies for $c_0$ and $|D|$ corresponding ordered pairs of the form $(c_0,d)$ then this implies $\exists\quad  |C|\cdot |D|$ unique ordered pairs, thus $|C\times D=|C||D|$ as the cartesian product is the set of all unique ordered pairs between two sets.
   \end{proof}
   \end{itemize}
 \item Determine which one of the following are equivalence relations. Justify.
   \begin{itemize}
    \item[a)] $x \sim y$ in $\R$ if $x \geq y$.
    \begin{enumerate}
      \item \textbf{Reflexivity}
      \begin{proof}
        Prove: $x\sim x$\\
        if $x\sim x$ then:
        \[
          x\geq x \rightarrow x=x
        \]
        So we have shown the relation satisfies Reflexivity.
      \end{proof}
      \item \textbf{Symmetry}
      \begin{proof}
        Prove $x \sim y \rightarrow y \sim x $\\
        if $x \sim y$ then $x \geq y$ from this inequality alone we cannot show that $y>x$ however it could be the case that $x=y$ in which case the statement $y \geq x$ would hold true staisfying symmetry. However, we cannot conclude if it is the case that $y=x$ given $x\geq y$ and thus we cannot prove the symmetritricity of the relation.

      \end{proof}\\
      \item \textbf{Transitivity}
      \begin{proof}
          Prove if $x\sim y$ and $y\sim z$ then $x \sim z $\\
          if $x \sim y$ then $x\geq y$, if $y \sim z$ then $ y\geq z $ then:
          \[
            x \geq y \geq z
          \]
          \[
            x\geq z
          \]
          Therefore $x\sim z$
      \end{proof}
      This relation is not an equivilance relation .
    \end{enumerate}
    \item[b)] $m \sim n$ in $\Z$ if $mn > 0$.
    \begin{enumerate}
      \item \textbf{Reflexivity}
      \begin{proof}
        Prove: $m\sim m$
        if $m\sim m $ then this implies \[
          m^2>0
        \]
        Consider the case of m =0, then this relation implies:
        \[
          0>0
        \]
        Which is not true, thus since Reflexivity fails for this relation this implies that the relation is not an equivilance relation.
      \end{proof}
      \item \textbf{Symmetry}
      \begin{proof}
        Prove $x \sim y \rightarrow y \sim x $\\
        If $x\sim y$ then :
        \[
          xy>0
        \]
        by commutatitivty of  integer multiplication this also gives:
    \[
      yx>0
    \]
    and thus we have that $y\sim x$ so y is related to x
      \end{proof}
      This relation is not an equivilance relation .
    \end{enumerate}
    \item[c)] $x \sim y$ in $\R$ if $|x-y| \leq 4$.
    \begin{enumerate}
      \item \textbf{Reflexivity}
      \begin{proof}
        Prove: $x\sim x$\\
        if $x \sim x $ then this implies :
        \begin{align*}
        &|x-x|\leq 4\\
        &|0|\leq 4\\
        &0\leq 4
        \end{align*}
      \end{proof}
      \item \textbf{Symmetry}
      \begin{proof}
        Prove $x \sim y \rightarrow y \sim x $\\
        if $x\sim y $ then :
        \[
          |x-y|\leq 4
        \]
        if $y\sim x$ then :
        \[
          |y-x|\leq 4
        \]
        For all x, y in the real numbers $|x-y|$ =$|y-x|$ therefore by this reason y is symmetric to x.
      \end{proof}
      \item \textbf{Transitivity}
      \begin{proof}
          Prove if $x\sim y$ and $y\sim z$ then $x \sim z $\\
          Consider the following hypothetical:
          given x,y,z $\in \R$ with x=4 y= 8, z= 12\\
          Then if $x \sim y$ this implies:
          \begin{align*}
            &|4-8|\leq 4 \\
            &4\leq 4
          \end{align*}
          \begin{align*}
            &|8-12|\leq ``4`` \\
            &4\leq 4
          \end{align*}
          \begin{align*}
            &|4-12|\leq 4 \\
            &8\leq 4
          \end{align*}
          so we have shown that the relation does not satisfy Transitivity.
      \end{proof}
    \end{enumerate}
    This relation is not an equivilance relation .
    \end{itemize}
\item Define the relation $(x_1, y_1) \sim (x_2,y_2)$ in $\R^2$ if $x_1^2 + y_1^2 = x_2^2 + y_2^2$. Show that this is an equivalence relation. Then describe the equivalence classes.
\begin{enumerate}
  \item \textbf{Reflexivity}
  \begin{proof}
    Prove: $x\sim x$\\
    if $x\sim x$ then we have that:
    \[
      x_1^2+y_1^2=x_1^2+y_1^2
    \]
    Which is true.
  \end{proof}
  \item \textbf{Symmetry}
  \begin{proof}
    Prove $x \sim y \rightarrow y \sim x $
    if $x\sim y$ then we have that:
    \[
      x_1^2+y_1^2=x_2^2+y_2^2
    \]
    which by the symmetric property of equality proves that $y\sim x $
  \end{proof}
  \item \textbf{Transitivity}
  \begin{proof}
      Prove if $x\sim y$ and $y\sim z$ then $x \sim z $\\
      if $x\sim y$ then \[
        x_1^2+y_1^2=x_2^2+y_2^2
      \]
      if $y\sim z $ then we have:
      \[
        x_2^2+y_2^2=x_3^2+y_3^2
      \]
      performing a substitution gives:
    \[
      x_1^2+y_1^2=x_3^2+y_3^2
    \]
    and thus $x\sim z$
  \end{proof}
\end{enumerate}
This allows us to conclude that this relation is an equivilance realtion. The equivilance classes of this relation take the following structure:
\[
  [a=(x_1,y_1)]=\{b\in \R: a\sim b\}=\{(x_2,y_2)\in \R^2:(x_1,y_1)\sim (x_2,y_2)\}
\]
so for some fixed ordered pair in $\R^2$ the equivilance class is the set of all other ordered pairs satisfying the relation. These relations consitute circles formed by the points related centered at $(x_1,y_1)$or rather $(x_i,y_i)$ for some arbitrary location in $\R^2$\\\\
In other words each equivilance class is a circle in $\R$ centered at the point you are drawing a relation with with radius: $x_2+y_2$
\item Define a relation on $\R^2 \setminus \{(0,0)\}$ by letting $(x_1, y_1) \sim (x_2, y_2)$ if there exists a nonzero $\lambda$ such that $(x_1,y_1) = (\lambda x_2, \lambda y_2)$.
Prove that $\sim$ defines an equivalence relation on $\R^2 \setminus \{(0,0)\}$. What are the equivalence classes ?

\begin{enumerate}
  \item \textbf{Reflexivity}
  \begin{proof}
    Prove: $x\sim x$\\
    If $x\sim x$ then we have the following relation:
    \[
      (x_1,y_1)=\lambda(x_2,y_2)
    \]
    which is true for the case of $\lambda$=1 proving Reflexivity.
  \end{proof}
  \item \textbf{Symmetry}
  \begin{proof}
    Prove $x \sim y \rightarrow y \sim x $\\
    If $x\sim y $ then:
    \[
      (x_1,y_1)=\lambda_1(x_2,y_2)
    \]
    Solving for $(x_2,y_2)$ we obtain:
    \[
      \frac{1}{\lambda_1}(x_1,y_1)=(x_2,y_2)
    \]
    So we have shown that for $\lambda_2=\frac{1}{\lambda_1}$ that $y\sim x $ and thus have proven symmetritricity.
  \end{proof}
  \item \textbf{Transitivity}
  \begin{proof}
      Prove if $x\sim y$ and $y\sim z$ then $x \sim z $\\
      If $x\sim y$ then :
      \[
        (x_1,y_1)=\lambda_1(x_2,y_2_)
      \]
      If $y\sim z$ then:
      \[
        (x_2,y_2)=\lambda_2(x_3,y_3)
      \]
      substituting for $(x_2,y_2)$:
      \[
        (x_1,y_1)=\lambda_1\lambda_2(x_3,y_3)
      \]
      so we have that $(x_1,y_1)\sim (x_3,y_3)$ for $\lambda_3=\lambda_1\lambda_2$ Proving Transitivity.
  \end{proof}
\end{enumerate}
 \end{enumerate}
The equivilance classes are the set of all scalar multiples of the vector formed by the original selection of $(x_1,y_1)$. This is to say that each equivilance class is a line in $\R$ where the elements on the line represent the span of the original vector chosen.


\end{document}
