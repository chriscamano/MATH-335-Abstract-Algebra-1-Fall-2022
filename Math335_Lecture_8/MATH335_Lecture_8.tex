\documentclass[11pt]{article}
\usepackage{amssymb,latexsym,amsmath,amsthm,graphicx, cite}
 \author{Chris Camano: ccamano@sfsu.edu}
 \title{MATH 335  lecture 5 }
 \date

\usepackage{mathptmx}
\usepackage{multirow}
\usepackage{float}
\restylefloat{table}
\hoffset=0in
\voffset=-.3in
\oddsidemargin=0in
\evensidemargin=0in
\topmargin=0in
\textwidth=6.5in
\textheight=8.8in
\marginparwidth 0pt
\marginparsep 10pt
\headsep 10pt

\theoremstyle{definition}  % Heading is bold, text is roman
\newtheorem{theorem}{Theorem}
\newtheorem{corollary}{Corollary}
\newtheorem{definition}{Definition}
\newtheorem{example}{Example}
\newtheorem{proposition}{Proposition}



\newcommand{\Z}{\mathbb{Z}}
\newcommand{\N}{\mathbb{N}}
\newcommand{\Q}{\mathbb{Q}}
\newcommand{\R}{\mathbb{R}}
\newcommand{\C}{\mathbb{C}}

\newcommand{\lcm}{\mathrm{lcm}}



\setlength{\parskip}{0cm}
%\renewcommand{\thesection}{\Alph{section}}
\renewcommand{\thesubsection}{\arabic{subsection}}
\renewcommand{\thesubsubsection}{\arabic{subsection}.\arabic{subsubsection}}
\bibliographystyle{amsplain}

%\input{../header}


\begin{document}
\maketitle
\definition Group: \\
A group is a set of objects and a binary relation: A  non empty set G with a binary operation (a,b)$\mapsto ab$ is called a group if :
\begin{enumerate}
  \item The binary operation is associative such that:
  \[
    ab(c)=a(bc)\quad \forall \quad a,b,c \in G
  \]
  \item
  The binary operation has to have an identity element denoted as 'e'. such that:
\[
  ae=ea=a \forall \quad a \in G
\]
  \item
  Every element has to have an inverse with respect to the binary operation. :
  \[
    \forall \quad a \in G , \exists\quad  b \in G : ab=ba=e
  \]
  let b be phrased as $a^{-1}$
\end{enumerate}




\end{document}
