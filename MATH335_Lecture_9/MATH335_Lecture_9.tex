\documentclass[11pt]{article}
\usepackage{amssymb,latexsym,amsmath,amsthm,graphicx, cite}
\usepackage{hyperref}
\hypersetup{
     colorlinks=true,
     linkcolor=blue,
     filecolor=blue,
     citecolor = black,
     urlcolor=blue,
     }
%\usepackage[sc]{mathpazo}
%\linespread{1.05}         % Palatino needs more leading (space between lines)
%\usepackage[T1]{fontenc}
\usepackage{mathptmx}
\usepackage{multirow}
\usepackage{float}
\restylefloat{table}
\hoffset=0in
\voffset=-.3in
\oddsidemargin=0in
\evensidemargin=0in
\topmargin=0in
\textwidth=6.5in
\textheight=8.8in
\marginparwidth 0pt
\marginparsep 10pt
\headsep 10pt

\theoremstyle{definition}  % Heading is bold, text is roman
\newtheorem{theorem}{Theorem}
\newtheorem{corollary}{Corollary}
\newtheorem{definition}{Definition}
\newtheorem{example}{Example}
\newtheorem{proposition}{Proposition}

\author{Chris Camano: ccamano@sfsu.edu}
\title{MATH 335  Lecutre 9 }
\date

\newcommand{\Z}{\mathbb{Z}}
\newcommand{\N}{\mathbb{N}}
\newcommand{\Q}{\mathbb{Q}}
\newcommand{\R}{\mathbb{R}}
\newcommand{\C}{\mathbb{C}}

\newcommand{\lcm}{\mathrm{lcm}}



\setlength{\parskip}{0cm}
%\renewcommand{\thesection}{\Alph{section}}
\renewcommand{\thesubsection}{\arabic{subsection}}
\renewcommand{\thesubsubsection}{\arabic{subsection}.\arabic{subsubsection}}
\bibliographystyle{amsplain}

%\input{../header}


\begin{document}
\maketitle
\definition: A group G is a non empty set with a binary operation $G\times G:\mapsto G$    $\quad (a,b)\in G^2\mapsto ab$
A bianry operation must satisfy the three following properties:
\begin{enumerate}
  \item Associativity: \\
\[
  a(bc)=(ab)c\quad  \forall a,b,c\in G
\]
The purpose of this property is to give access to statements such as $abc$ without concern for ordering in operation composition
\item
The existence of the identity element e in G\\
Such that:
\[
  ae=ea=a \quad \forall a,e \in G
\]
\item
For all elements in G there exists an inverse uder the bianry oepration such that :
\[
  a(a^{-1})=(a^{-1})a=e
\]
\end{enumerate}
\textbf{Common Examples}:\\
Let G=$\{\Z,+\}$ This group has an identity element, inverse and Associativity over addition. \\
Let G=$\{\R/\{0\},\cdot\}$: Associativity over multiplication, idenity over 1, inverse would be the reciprocal of any element inG
\sect{A new important Group}\\
Let $n\in \Z^+$ Recall the equivilance realation on $\Z$ defined by:
\[
  a \sim b \rightarrow n|a-b
\]
or rather:
\[
  a\equiv b \mod n
\]
The set of equivilance classes of this equivilance relation is denoted as: $\Z_n  $
\[
  \Z_n=\left\{ [0],[1],[2],[3],...,[n-1] \right\}
\]
\[
  |\Z_n|=n
\]
We define the following binary operation on $\Z_n$: We cal this operation addition modulo n:\\\\
\[
  [a]\circ [b]:=[(a+b) \mod n ]
\]
with addition we then say:
\[
[a\mod n]\textbf{+} [b\mod n]:=[(a+b) \mod n ]
\]\\\\
\definition Well defined: Does it depend on the names of the objects being related.\\\\
Proof of well definition of operation of equivilance class addition: \\
Suppose:
\[
  [a]=[a^*],[b]=[b^*]
\]
We wish to show that :
\[
  [a]+[b]=[a^*]+[b^*]
\]
$[a]=[a^*]\right arrow n|a-a^*, [b]=[b^*]\rightarrow n|b-b^*$
\begin{align*}
  &[a]+[b]=[a^*]+[b^*]\\\\
  &[a+b]=[a^*+b^*]\\\\
\end{align*}
We need to show now that:
\[
  n|a+b-a^*+b^*
\]
\begin{align*}
    &n|[a+b]-[a^*+b^*]\\
    &n|a-a^*+b-b^*\\
    &n|n(k)+n(l),k,l\in \Z\\
    &n|n(k+l)
\end{align*}
We now prove this is a group\\\\
\begin{enumerate}
  \item
  \begin{align*}
    &[a]+([b]+[c])=([a]+[b])+[c]\\
    &[a]+([b+c])=([a+b])+[c]\\
    &[a+(b+c)]=[(a+b)+c]\\
    &[a+b+c]=[a+b+c]
  \end{align*}
  \item Identity element [0]
  \item Inverse : [n-a]$\sim[-a]$\\\\
So we have proven that $\Z_n$ under addition is a group.
\end{enumerate}
\definition A group G is called abelian or "commutative" if the binary operation on the group does not depend on the order of the operands. This is to say:
\[
  ab=ba \quad \forall a,b \in G
\]
\sect{Example of non nonabelian Group}:
The set of invertible sqaure matrices under multiplication is nonabelian. This is called the order n general linear group.
\end{document}
