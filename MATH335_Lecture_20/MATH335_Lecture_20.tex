\documentclass[11pt]{article}
\usepackage{amssymb,latexsym,amsmath,amsthm,graphicx, cite}
\author{Chris Camano: ccamano@sfsu.edu}
\title{MATH 335  Lecture 20 }
\date

\usepackage[many]{tcolorbox}
\tcbset{breakable}
\usepackage{mathptmx}
\usepackage{multirow}
\usepackage{float}
\restylefloat{table}
\hoffset=0in
\voffset=-.3in
\oddsidemargin=0in
\evensidemargin=0in
\topmargin=0in
\textwidth=6.5in
\textheight=8.8in
\marginparwidth 0pt
\marginparsep 10pt
\headsep 10pt

\theoremstyle{definition}  % Heading is bold, text is roman
\newtheorem{theorem}{Theorem}
\newtheorem{corollary}{Corollary}
\newtheorem{defn}{Definition}
\newtheorem{example}{Example}
\newtheorem{proposition}{Proposition}
\usepackage{listings}
\usepackage{xcolor}
\lstset { %
    language=C++,
    backgroundcolor=\color{black!5}, % set backgroundcolor
    basicstyle=\footnotesize,% basic font setting
}


\newcommand{\Z}{\mathbb{Z}}
\newcommand{\N}{\mathbb{N}}
\newcommand{\Q}{\mathbb{Q}}
\newcommand{\R}{\mathbb{R}}
\newcommand{\C}{\mathbb{C}}
\newcommand{\lcm}{\mathrm{lcm}}
\setlength{\parskip}{0cm}
%\renewcommand{\thesection}{\Alph{section}}
\renewcommand{\thesubsection}{\arabic{subsection}}
\renewcommand{\thesubsubsection}{\arabic{subsection}.\arabic{subsubsection}}
\bibliographystyle{amsplain}

%\input{../header}
\newcommand{\bigline}{\\\noindent\makebox[\linewidth]{\rule{\textwidth}{0.4pt}}\\}

\newcommand{\block}[2]{\begin{tcolorbox}[title={#1}]{#2}\end{tcolorbox}}
\begin{document}
\maketitle

\block{Lagranges theorem and associated concequences}{
Let G be a finite group and H a subgroup of G, then the index of H in G denoted:
\[
  [G:H]
\] Which is equivilant to the number of left cosets of H in G
\[
  [G:H]=\frac{|G|}{|H|}
\]
\\
In particular the order of H divides the order of G.
\begin{enumerate}
  \item If G is a finte group and we pick any element in $g\inG$ then $|g|||G|$\\
  \begin{proof}
    The order of an element g is equal to the number of elements in the cyclic subgroup generated by g this is to say:
    \[
      |g|=|<g>|
    \]
    So by lagranges theorem :
    \[
      |<g>|||G|
    \]
  \end{proof}
  \item In a finite group G with $|G|=n$, $g^n=e$ for all $g\in G$\\
  \begin{proof}
    let $|g|=d$ then by the previous proof we know that $d|n$
    \[
      g^n=(g^d)^\frac{n}{d}=e^\frac{n}{d}=e
    \]
    Since n over d is an integer by Lagranges theorem:
  \end{proof}
  \item If G is a group with order p where p is a prime number, then G must be a cylic\\
  \begin{proof}
    $p\geq 2$ so there is at least one non identity element, g$\new e$ Consider the subgroup generated by this element:
    \[
      H=<g>
    \]
    Lagranges theorem gives us $|H|||G|$ but the order of G is a prime number, therefore:
    \[
      |H|=1 \text { or }p
    \]
    but we said that it cannot be 1 meaning the only possibility is that H has order p so
    \[
      H=<g>=G
    \]
    If the order of G is prime any non identity element is a generator
  \end{proof}
\end{enumerate}
}
\block{\textbf{Euler's theorem}}{
\\
Suppose a,n , n>0 are integers that are coprime.
\\
Then :
\[
  a^{\phi(n)}\equiv 1 \mod n
\]
\begin{proof}
  $$U(n)=\{[m]:gcd(n,m)=1\}$$
  $$|U(n)|=\phi(n)$$
  Then by Collary two if we take a finite group element and raise it to the order we get identity. We can use this since gcd(a,n)=1\\
  \[
    [a]^{|U(n)|}=[1]
  \]
  \[
    [a]^{\phi(n)}=[1]
  \]
  \[
    a^{\phi(n)}\equiv 1 \mod n
  \]
\end{proof}
}
\block{Fermat's Little Theorem}{
If a prime p does not divide a, this is to say p and a are coprime, then \[
  a^{p-1}\equiv 1 \mod p
\]
\begin{proof}
  $a^{\phi(p)}=a^{p-1}\equiv 1 \mod p 
\end{proof}


}
\end{document}
