\documentclass[11pt]{article}
\usepackage{amssymb,latexsym,amsmath,amsthm,graphicx, cite}
\author{Chris Camano: ccamano@sfsu.edu}
\title{MATH 335  lecture 18 }
\date

\usepackage[many]{tcolorbox}
\tcbset{breakable}
\usepackage{mathptmx}
\usepackage{multirow}
\usepackage{float}
\restylefloat{table}
\hoffset=0in
\voffset=-.3in
\oddsidemargin=0in
\evensidemargin=0in
\topmargin=0in
\textwidth=6.5in
\textheight=8.8in
\marginparwidth 0pt
\marginparsep 10pt
\headsep 10pt

\theoremstyle{definition}  % Heading is bold, text is roman
\newtheorem{theorem}{Theorem}
\newtheorem{corollary}{Corollary}
\newtheorem{defn}{Definition}
\newtheorem{example}{Example}
\newtheorem{proposition}{Proposition}



\newcommand{\Z}{\mathbb{Z}}
\newcommand{\N}{\mathbb{N}}
\newcommand{\Q}{\mathbb{Q}}
\newcommand{\R}{\mathbb{R}}
\newcommand{\C}{\mathbb{C}}
\newcommand{\lcm}{\mathrm{lcm}}
\setlength{\parskip}{0cm}
%\renewcommand{\thesection}{\Alph{section}}
\renewcommand{\thesubsection}{\arabic{subsection}}
\renewcommand{\thesubsubsection}{\arabic{subsection}.\arabic{subsubsection}}
\bibliographystyle{amsplain}

%\input{../header}
\newcommand{\bigline}{\\\noindent\makebox[\linewidth]{\rule{\textwidth}{0.4pt}}\\}

\newcommand{\block}[2]{\begin{tcolorbox}[title={#1}]{#2}\end{tcolorbox}}
\begin{document}
\maketitle


\textbf{Refresher of last lecture}\\
\textbf{Left Coset:}\\
 Let G be a group and H be a subgroup of G. For $g\inG$ the left coset of H with representative g is the set which is denoted as follows:
\[
  gH=\{gh:h\in H\}
\]
\textbf{Right Coset:}\\
 Let G be a group and H be a subgroup of G. For $g\inG$ the left coset of H with representative g is the set which is denoted as follows:
\[
  Hg=\{hg:h\in H\}
\]
Cosets are a means of relating the size of the subgroup H to the size of the group G, more to follow:\\
\block{Original Example}{
Let: \[
  G=S_3=\{e,\tau,\tau^2,\sigma_1,\sigma_2,\sigma_3\}
\]
\[
  H=<\tau>=\{e,\tau,\tau^2\}
\]
\[
  H=<\sigma_1>=\{e,\sigma_1\}
\]
\[
\tau=
  \begin{bmatrix}
    1&2&3\\2&3&1
  \end{bmatrix}
\]
\[
\sigma_1=
  \begin{bmatrix}
    1&2&3\\2&1&3
  \end{bmatrix}
\]
}\\\\
Compute all left cosets of H. \\
Compute all left cosets of K and all right cosets of K\\
\\
For the left cosets of H we obtain only two unique cosets \\
The elements of a cosets generate the same coset if used as a representative. \\\\
A coset is a set of its own representatives\\
\block{Coset observations}{
\begin{enumerate}
  \item
  $$
  g_1H=g_2H \iff g_2 \in g_1H
  $$
  \item
  \[
    \frac{|G|}{|H|}=\text { number of Unique  left cosets of H }=\text { number of Unique right cosets of H }
  \]
  \item
  \[
    |H|||G|
  \]
  \item
  In general  left coset is not equal to the right coset.
  \item
  order of each coset is the order of H.
  \item The distinct cosets partition the group\\

\end{enumerate}




}
\end{document}
