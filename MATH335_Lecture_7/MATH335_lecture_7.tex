\documentclass[11pt]{article}
\usepackage{amssymb,latexsym,amsmath,amsthm,graphicx, cite}
 \author{Chris Camano: ccamano@sfsu.edu}
 \title{MATH 335  lecture 7 }
 \date

\usepackage{mathptmx}
\usepackage{multirow}
\usepackage{float}
\restylefloat{table}
\hoffset=0in
\voffset=-.3in
\oddsidemargin=0in
\evensidemargin=0in
\topmargin=0in
\textwidth=6.5in
\textheight=8.8in
\marginparwidth 0pt
\marginparsep 10pt
\headsep 10pt

\theoremstyle{definition}  % Heading is bold, text is roman
\newtheorem{theorem}{Theorem}
\newtheorem{corollary}{Corollary}
\newtheorem{definition}{Definition}
\newtheorem{example}{Example}
\newtheorem{proposition}{Proposition}



\newcommand{\Z}{\mathbb{Z}}
\newcommand{\N}{\mathbb{N}}
\newcommand{\Q}{\mathbb{Q}}
\newcommand{\R}{\mathbb{R}}
\newcommand{\C}{\mathbb{C}}

\newcommand{\lcm}{\mathrm{lcm}}



\setlength{\parskip}{0cm}
%\renewcommand{\thesection}{\Alph{section}}
\renewcommand{\thesubsection}{\arabic{subsection}}
\renewcommand{\thesubsubsection}{\arabic{subsection}.\arabic{subsubsection}}
\bibliographystyle{amsplain}

%\input{../header}


\begin{document}

\maketitle
\defn equivilance relation\\
An equivilance relation$\sim$ on a set X is a relation where three properties are held:
\begin{enumerate}
  \item $x \sim x \quad \forall x\in X$ Reflexive
  \item $s\sim y \iff y\sim x \quad \forall x \in X$ Symmetric
  \item $x\sim y \land y\sim z $ implies $x\sim z $ Transitivity.
\end{enumerate}
equivilance realations are used to collect elements in a set for some intended purpose. It partitions the set by creating classifiers for the elements of the set. \\\\
Fundemental example:\\

We first start by fixing some m$\in \Z^+$ (this works for any positive integer). We define a realtion on all integers $\Z$ such that: \\
Given two integers a,b $a\sim b \iff n |a-b$ in other words: $a \equiv b \mod n $.

\begin{enumerate}
  \item $x \sim x \quad \forall x\in X$ Reflexive
  \begin{proof}
    $a\equiv b mod n$ since $n|a-a$
  \end{proof}
  \item $s\sim y \iff y\sim x \quad \forall x \in X$ Symmetric
  \begin{proof}
    $n|(x-y)$ so $n|-(x-y)$ so $n|y-x$
  \end{proof}
  \item $x\sim y \land y\sim z $ implies $x\sim z $
   Transitivity.
   \begin{proof}
     $n|a-b$ and $n|b-c$ so $n|a-b+b=n|a-c$ so $a\sim c$
   \end{proof}
\end{enumerate}
\defn Equivilance classes\\
Let X be a set, and $\sim$ be an equivilance realtion on X. Let $x\in X$ Then:
\[
  [x]=\{y\in X: y\sim x\}
\]
This set is called the equivilance class of x.The equivilance class of x always contains x.
\\
The union of all equivilance classed mod n paritions the interges. :
\[
  \cup_{i=1}^\infty[x_i]=\Z
\]
Also \[
  x_i\cap x_j=\empty , i\neq j
\]


\end{document}
