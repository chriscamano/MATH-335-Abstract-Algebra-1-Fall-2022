\documentclass[11pt]{article}
\usepackage{amssymb,latexsym,amsmath,amsthm,graphicx, cite}
\author{Chris Camano: ccamano@sfsu.edu}
\title{MATH 325 Lecture 19 }
\date

\usepackage[many]{tcolorbox}
\tcbset{breakable}
\usepackage{mathptmx}
\usepackage{multirow}
\usepackage{float}
\restylefloat{table}
\hoffset=0in
\voffset=-.3in
\oddsidemargin=0in
\evensidemargin=0in
\topmargin=0in
\textwidth=6.5in
\textheight=8.8in
\marginparwidth 0pt
\marginparsep 10pt
\headsep 10pt

\theoremstyle{definition}  % Heading is bold, text is roman
\newtheorem{theorem}{Theorem}
\newtheorem{corollary}{Corollary}
\newtheorem{defn}{Definition}
\newtheorem{example}{Example}
\newtheorem{proposition}{Proposition}



\newcommand{\Z}{\mathbb{Z}}
\newcommand{\N}{\mathbb{N}}
\newcommand{\Q}{\mathbb{Q}}
\newcommand{\R}{\mathbb{R}}
\newcommand{\C}{\mathbb{C}}
\newcommand{\lcm}{\mathrm{lcm}}
\setlength{\parskip}{0cm}
%\renewcommand{\thesection}{\Alph{section}}
\renewcommand{\thesubsection}{\arabic{subsection}}
\renewcommand{\thesubsubsection}{\arabic{subsection}.\arabic{subsubsection}}
\bibliographystyle{amsplain}

%\input{../header}
\newcommand{\bigline}{\\\noindent\makebox[\linewidth]{\rule{\textwidth}{0.4pt}}\\}

\newcommand{\block}[2]{\begin{tcolorbox}[title={#1}]{#2}\end{tcolorbox}}
\begin{document}
\maketitle
Opening Notes exam is on thursday November third, study guide coming friday evening or tomorrow. Short homework will be due tuesday, Submitted on i learn tuesday at 10am. \\
In class eam will mostly be cyclic groups the take home will  be on cosets.
\block{A return to the properties of Cosets}{
  \item \proposition Let G be a group and H a subgroup of G. Then :
\[
  g_1H=g_2H \iff g_2\in H
\]
A coset generates its own representatives. \\




}
\begin{proof}
  $\newline $
  \textbf{ $ g_1H=g_2H \rightarrow g_2\in g_1H$ }\\
  Since H is a subgroup when we compute:
  $
    g_2H
  $
  We are guarenteed to have at minimum $g_2\in g_2H$ since $g_2\circ e =g_2$
  So $g_2\in g_2H$
\end{proof}
\begin{proof}
  $\newline$
  \textbf{$  g_1H=g_2H \leftarrow g_2\in H$ }\\
  Given $g_2\in g_1H$ This means that $g2=g_1\bar{h}, \bar{h}\in H$ :
  Consider the set $g_2H=\{g_2h,h\in H\}$\\
  but we know that $g_2=g_1\bar{h}$ so :
  \[
    g_2H=\{g_2h,h\in H\}=\{g_1\bar{h}h:h\in H\}
  \]
  Note that since $\bar{h},h\in H$ that $\bar{h}h\in H$ and it follows that:
  \[
    g_1\bar{h}h\in H
  \]
  Thus:
  \[
    g_2H\subset g_1H
  \]
  We can also redefine $g_2$ as follows;
  \[
    g_2=g_1\bar{h}\rightarrow g_2\bar{h}^{-1}=g_1
  \]
  So:
  \[
    g_1H=\{g_1h:h\in H\}=\{g_2\bar{h}^{-1}h:h\in H}\subset g_2H
  \]
\end{proof}
\block{Theorem}{
 Let G be a group and H a subgroup of G then the distinct left cosets of H in G partition G.
}
Implications:
\begin{enumerate}
  \item The union of all left cosets of H is G
  \item if $g_1H\neq g_2H$ then $g_1H \cap g_2H=\empty$
\end{enumerate}
\begin{proof}
  $\newline$
  \textbf{The union of all left cosets of H is G}\\
  Given an element: $g\in G$ we show that g is some left coset of H g is in $gH$ since g composed with identity guarentees us the existence of a coset with g. \\
\end{proof}
\begin{proof}
  $\newline$
  \textbf{if $g_1H\neq g_2H$ then $g_1H \cap g_2H=\empty$}\\
  We show that if the intersection of $g_1H\cap g_2H$ is not empty then $g_1H=g_2H$ this is the contrapositive of our original argument. Since the intersection is non empty there must exist an element in the intersection. Let g be an element in the intersection of the two cosets\\
  Then:
  \[
    g=g_1h_1, h_1\in H \quad g=g_2h_2, h_2\in H
  \]
  Which implies:
  \[
    g_1h_1=g_2h_2
  \]
  \[
    g_2=g_1h_1h_2^{-1},
  \]
  but: $h_1h_2^{-1}\in H$ So we have that $g_2\in g_1H$ and by the Previous proof we have: $g_2\in g_1H$
\end{proof}
If you have an infinite group there are infinite representatives for each coset. \\
You can have a finite quantity of cosets for an infinite group and infinite subgroup\\
Example G=$\Z$ $H=<n>$ The cosets are the equivilance classes \\
The cosets are the elements of: $\Z_n$ \\
The set of cosets does not always form a new group like in this example. \\
\block{Proposition}{Let G be a group and H a subgroup of G, Then for any $g\in G$ there is a bijection between the subgroup H and the left coset gH. In particular, if H is a finite subgroup, then the number of elements in H and the number of elements in its left coset are the same, this is to say:
\[
  |H|=|gH|
\]}
\begin{proof}
  $\newline$
  Let $\phi:H\mapsto gH$ be : $\phi(h)=gh, h\in H, $.
  We begin by proving surjectivity: Given an element of gH,: gh for some h. Then $\phi(h)=gh$ \\
  Next is injectivity: Suppose $\phi(h_1)=\phi(h_2)$ Show $h_1=h_2$

  \[
    \phi(h_1)=gh_1\quad \phi(h_2)=gh_2
  \]
  Then by our assumption:
  \[
    gh_1=gh_2
  \]
  but we can left multiply by g inverse since we are in a group thus :
  \[
    h_1=h_2
  \]
\end{proof}
\block{Major theorem:Lagrange's theorem}{
Let G be a finite group.and H be a subgroup, The the number of distinct left cosets of H in G is equal to:
\[
  \frac{|G|}{|H|}
\]
in particular: $|H|||G|$
}
\begin{proof}
  Distinct left cosets of H partition G. Every left Coset has as many elements as in H, This to say :
  \[
    \forall \text{ cosets } c_i, |c_i|=|H|
  \]
  Thus $|G|=|H|$ times the number of distinct left cosets of H\\
\end{proof}
\block{Colloary}{
let G be a finite group and let a$\in G$ Then $|a||G|$ \\
\[
  |a|=|<a>|
\]
Then by Lagrange's theorem $|<a>|||G|$
}
\block{Colloray}{

If $|G|=p$ then G must be a cyclic group
}
\begin{proof}
  Pick any non identity element in G a . Consider $H=<a>$ by lagranges theorem $|H|||G|, |H||p$ so the order of H must be p. but the order of the group and subgroup is identical so $G=<a>$
\end{proof}
\end{document}
