\documentclass[11pt]{article}
\usepackage{amssymb,latexsym,amsmath,amsthm,graphicx, cite}
\usepackage{hyperref}
\hypersetup{
     colorlinks=true,
     linkcolor=blue,
     filecolor=blue,
     citecolor = black,
     urlcolor=blue,
     }
%\usepackage[sc]{mathpazo}
%\linespread{1.05}         % Palatino needs more leading (space between lines)
%\usepackage[T1]{fontenc}
\usepackage{mathptmx}
\usepackage{multirow}
\usepackage{float}
\restylefloat{table}
\hoffset=0in
\voffset=-.3in
\oddsidemargin=0in
\evensidemargin=0in
\topmargin=0in
\textwidth=6.5in
\textheight=8.8in
\marginparwidth 0pt
\marginparsep 10pt
\headsep 10pt

\theoremstyle{definition}  % Heading is bold, text is roman
\newtheorem{theorem}{Theorem}
\newtheorem{corollary}{Corollary}
\newtheorem{definition}{Definition}
\newtheorem{example}{Example}
\newtheorem{proposition}{Proposition}

\author{Chris Camano: ccamano@sfsu.edu}
\title{MATH 335  Homework 3 }
\date

\newcommand{\Z}{\mathbb{Z}}
\newcommand{\N}{\mathbb{N}}
\newcommand{\Q}{\mathbb{Q}}
\newcommand{\R}{\mathbb{R}}
\newcommand{\C}{\mathbb{C}}

\newcommand{\lcm}{\mathrm{lcm}}



\setlength{\parskip}{0cm}
%\renewcommand{\thesection}{\Alph{section}}
\renewcommand{\thesubsection}{\arabic{subsection}}
\renewcommand{\thesubsubsection}{\arabic{subsection}.\arabic{subsubsection}}
\bibliographystyle{amsplain}

%\input{../header}


\begin{document}
\maketitle
%\homework{}{Homework IV}

\begin{enumerate}
%%%%%%%%%%%%%%%%%%%%%%%%%%%%%%%%%%%%%%%%%%%%%
\item Deterimine which one of the following sets is a group under addition:
Addition is a binary operator by the proof provided in class, therefore we will validate the properties of each group accepting that addition is binary operator in these contexts.\\\\
only 1 a is a group associativity fails.
\\\\

  \begin{itemize}
    \item[a)] the set of rational numbers in lowest terms whose denominators are odd
    \begin{proof}
      \\
      Let G be the following set:
      \[
        G=\left\{\frac{m}{n},n=2k+1,n\neq0,k,m\in \Z\right\}
      \]
      \begin{enumerate}
        \item Proof of associativity over operator
        let a,b,c$\in$G where:
        \[
          a=\frac{a_m}{a_n}\quad b=\frac{b_m}{b_n}\quad
          c=\frac{c_m}{c_n}
        \]
        \begin{align*}
          &\left(\frac{a_m}{a_n}+\frac{b_m}{b_n}\right)+\frac{c_m}{c_n}=\frac{a_m}{a_n}+\left(\frac{b_m}{b_n}+\frac{c_m}{c_n}\right)\\\\
          &\frac{a_mb_n+b_ma_n}{a_nb_n}+\frac{c_m}{c_n}=\frac{a_m}{a_n}+\left(\frac{b_mc_n+c_mb_n}{b_nc_n}\right)\\\\
          &\frac{c_n(a_mb_n+b_ma_n)+c_ma_nb_n}{a_nb_nc_n}=\frac{a_n(b_mc_n+c_mb_n)+a_mb_nc_n}{a_nb_nc_n}\right)\\\\
          &\frac{c_na_mb_n+c_nb_ma_n+c_ma_nb_n}{a_nb_nc_n}=\frac{c_na_mb_n+c_nb_ma_n+c_ma_nb_n}{a_nb_nc_n}\right)
        \end{align*}
        The denominator $a_nb_nc_n$ is the product of three odd numbers which is also odd preserving the construction of G.
        \item Proof of existence of identity element\\
        The Identity element for this set is the element: $\frac{0}{1}=0$
        \item  Proof of existence of inverse of operator\\
        For all g$\in G$ the additive inverse is the element $-g$
      \end{enumerate}
    \end{proof}
    %%%%%%%%%%%%%%%%%%%%%%%%%%%%%%%%%%%%%%%%%%%%%
    \item[b)] the set of rational numbers in lowest terms whose denominators are even
    \begin{proof}
      This set is not closed under the binary operator, consider the following example:
      \[
        a=\frac{1}{6}\quad b=\frac{1}{6}
      \]
      under addition this yields:
      \[
        \frac{1}{6}+\frac{1}{6}=\frac{1}{3}\notin G
      \] Thus since the binary operator fails under closure, this set and operation do \textbf{not} form a group
    \end{proof}
    %%%%%%%%%%%%%%%%%%%%%%%%%%%%%%%%%%%%%%%%%%%%%
    \item[c)] the set of rational numbers of absolute value $<1$
    \\
    Let G be the following set:
    \[
      G=\left\{\frac{m}{n},n\neq0,n,m\in \Z, |\frac{m}{n}|<1\right\}
    \]
    \begin{proof}\\
      The binary operator of addition fails for this set when dealing with elements whose sum is greater than one. There are infinite examples of this, let us pick an apparent one:
      \[
        a=\frac{1}{2}\quad b=\frac{1}{2}
      \]
      \[
        a+b=\frac{1}{2}+\frac{1}{2}=1\notin G
      \]
      Due to the fact that the binary operator fails for closure this operator and set G do \textbf{not} form a group
    \end{proof}
    %%%%%%%%%%%%%%%%%%%%%%%%%%%%%%%%%%%%%%%%%%%%%
     \item[d)] the set of rational numbers of
     absolute value $\geq 1$ together with $0$.
     \\
     Let G be the following set:
     \[
       G=\left\{\frac{m}{n},n\neq0,k,m\in \Z, |\frac{m}{n}|>1\right\}\cup \{0\}
     \]
     \begin{proof}
       Due to the construction of G taking the absolute value over each of its elements this opens weaknesses in the closure of the binary operator. We can easily pick negative values less than -1 and show that under addition with a positive element that we can arrive at a value less than 1 when the difference between our selection of a and b is satisfactory:
       \[
         a=\frac{-5}{3}\quad b=\frac{3}{2}
       \]
       \[
         a+b=\frac{-5}{3}+\frac{3}{2}=\frac{-1}{6}\notin G
       \]
       Thus, again, since we fail under closure of the binary operator the set G does \textbf{not} form a group under the operator of addition
     \end{proof}
%%%%%%%%%%%%%%%%%%%%%%%%%%%%%%%%%%%%%%%%%%%%%
   \end{itemize}
%%%%%%%%%%%%%%%%%%%%%%%%%%%%%%%%%%%%%%%%%%%%%
 \item Let $G = \{x \in \R \, : \, 0 \leq x < 1\}$ and for $x,y \in G$ let $x \cdot y$ be the fractional part of $x + y$ (i.e. $ x \cdot y = x + y - [x+y]$ where $[a]$ is the greatest
   integer less than or equal to $a$). Prove that $\cdot$ is a binary operation on $G$ and that $G$ is a group.
   \begin{proof}Proof that $\cdot$ is a binary operator

   \end{proof}
   \begin{proof} Proof that G is a group
     \begin{enumerate}
       \item Proof of associativity over operator
       \item Proof of existence of identity element
       \item  Proof of existence of inverse of operator
     \end{enumerate}
   \end{proof}
%%%%%%%%%%%%%%%%%%%%%%%%%%%%%%%%%%%%%%%%%%%%%
\item Let $G = \{z \in \C \, : \, z^n = 1 \mbox{  for some nonnegative integer  } n\}$. Prove that $G$ is a group under multiplication (called the groups of {\it roots of unity} in $\C$).
\begin{proof}
  \begin{enumerate}
     \item Proof of Closure under operator:
     let :
     \[
       a=z_1^a=1\quad b=z_2^b=1
     \]
     Take $z_{1,2}=z_1z_2$ and $n_{1,2}=ab$
     \begin{align*}
       z_{1,2}^{ab}=(z_1z_2)^{ab}=z_1^{ab}z_2^{ab}=(z_1^a)^b(z_2^b)^a=1^b1^a=1
     \end{align*}
    \item Proof of associativity over operator
    \item Proof of existence of identity element
    \item  Proof of existence of inverse of operator
  \end{enumerate}
\end{proof}
%%%%%%%%%%%%%%%%%%%%%%%%%%%%%%%%%%%%%%%%%%%%%
\item Let $G = \{ a + b\sqrt{2} \, : \, a, b \in \Q\}$.
  \begin{itemize}
  \item[a)] Prove that $G$ is a group under addition.
  \begin{proof}
    \begin{enumerate}
       \item Proof of Closure under operator
      \item Proof of associativity over operator\\
      Let
      \[
        a=a_1+b_1\sqrt{2}\quad b=a_2+b_2\sqrt{2})\quad c= a_3+b_3\sqrt{2}
      \]

      \begin{align*}
        &(a+b)+c=a+(b+c)\\
        &(a_1+b_1\sqrt{2}+a_2+b_2\sqrt{2})+a_3+b_3\sqrt{2}=a_1+(b_1\sqrt{2}+a_2+b_2\sqrt{2}+a_3+b_3\sqrt{2})\\
        &a_1+b_1\sqrt{2}+a_2+b_2\sqrt{2}+a_3+b_3\sqrt{2}=a_1+b_1\sqrt{2}+a_2+b_2\sqrt{2}+a_3+b_3\sqrt{2}
      \end{align*}
      \item Proof of existence of identity element

      \item  Proof of existence of inverse of operator
    \end{enumerate}
  \end{proof}
  %%%%%%%%%%%%%%%%%%%%%%%%%%%%%%%%%%%%%%%%%%%%%
   \item[b)] Prove that the nonzero elements of $G$ are a group under multiplication [``Rationalize the denominators'' to find the inverses].
   \begin{proof}
     \begin{enumerate}
       \item Proof of Closure under operator
       \item Proof of associativity over operator
       \item Proof of existence of identity element
       \item  Proof of existence of inverse of operator
     \end{enumerate}
   \end{proof}
  \end{itemize}
  %%%%%%%%%%%%%%%%%%%%%%%%%%%%%%%%%%%%%%%%%%%%%
 \end{enumerate}



\end{document}
