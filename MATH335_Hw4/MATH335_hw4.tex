\documentclass[11pt]{article}
\usepackage{amssymb,latexsym,amsmath,amsthm,graphicx, cite}
\usepackage{hyperref}
\hypersetup{
     colorlinks=true,
     linkcolor=blue,
     filecolor=blue,
     citecolor = black,
     urlcolor=blue,
     }
%\usepackage[sc]{mathpazo}
%\linespread{1.05}         % Palatino needs more leading (space between lines)
%\usepackage[T1]{fontenc}
\usepackage{mathptmx}
\usepackage{multirow}
\usepackage{float}
\restylefloat{table}
\hoffset=0in
\voffset=-.3in
\oddsidemargin=0in
\evensidemargin=0in
\topmargin=0in
\textwidth=6.5in
\textheight=8.8in
\marginparwidth 0pt
\marginparsep 10pt
\headsep 10pt

\theoremstyle{definition}  % Heading is bold, text is roman
\newtheorem{theorem}{Theorem}
\newtheorem{corollary}{Corollary}
\newtheorem{definition}{Definition}
\newtheorem{example}{Example}
\newtheorem{proposition}{Proposition}

\author{Chris Camano: ccamano@sfsu.edu}
\title{MATH 335  Homework 3 }
\date

\newcommand{\Z}{\mathbb{Z}}
\newcommand{\N}{\mathbb{N}}
\newcommand{\Q}{\mathbb{Q}}
\newcommand{\R}{\mathbb{R}}
\newcommand{\C}{\mathbb{C}}

\newcommand{\lcm}{\mathrm{lcm}}



\setlength{\parskip}{0cm}
%\renewcommand{\thesection}{\Alph{section}}
\renewcommand{\thesubsection}{\arabic{subsection}}
\renewcommand{\thesubsubsection}{\arabic{subsection}.\arabic{subsubsection}}
\bibliographystyle{amsplain}

%\input{../header}


\begin{document}
\maketitle
%\homework{}{Homework IV}

\begin{enumerate}
%%%%%%%%%%%%%%%%%%%%%%%%%%%%%%%%%%%%%%%%%%%%%
\item Deterimine which one of the following sets is a group under addition:
Addition is a binary operator by the proof provided in class, therefore we will validate the properties of each group accepting that addition is binary operator in these contexts.
  \begin{itemize}
    \item[a)] the set of rational numbers in lowest terms whose denominators are odd
    \begin{proof}
      \begin{enumerate}
        \\
        \item Proof of associativity over operator
        \item Proof of existence of identity element
        \item  Proof of existence of inverse of operator
      \end{enumerate}
    \end{proof}
    %%%%%%%%%%%%%%%%%%%%%%%%%%%%%%%%%%%%%%%%%%%%%
    \item[b)] the set of rational numbers in lowest terms whose denominators are even
    \begin{proof}\\
      \begin{enumerate}
        \item Proof of associativity over operator
        \item Proof of existence of identity element
        \item  Proof of existence of inverse of operator
      \end{enumerate}
    \end{proof}
    %%%%%%%%%%%%%%%%%%%%%%%%%%%%%%%%%%%%%%%%%%%%%
    \item[c)] the set of rational numbers of absolute value $<1$
    \begin{proof}\\
      \begin{enumerate}
        \item Proof of associativity over operator
        \item Proof of existence of identity element
        \item  Proof of existence of inverse of operator
      \end{enumerate}
    \end{proof}
    %%%%%%%%%%%%%%%%%%%%%%%%%%%%%%%%%%%%%%%%%%%%%
     \item[d)] the set of rational numbers of
     absolute value $\geq 1$ together with $0$.
     \begin{proof}
       \begin{enumerate}
         \item Proof of associativity over operator
         \item Proof of existence of identity element
         \item  Proof of existence of inverse of operator
       \end{enumerate}
     \end{proof}
%%%%%%%%%%%%%%%%%%%%%%%%%%%%%%%%%%%%%%%%%%%%%
   \end{itemize}
%%%%%%%%%%%%%%%%%%%%%%%%%%%%%%%%%%%%%%%%%%%%%
 \item Let $G = \{x \in \R \, : \, 0 \leq x < 1\}$ and for $x,y \in G$ let $x \cdot y$ be the fractional part of $x + y$ (i.e. $ x \cdot y = x + y - [x+y]$ where $[a]$ is the greatest
   integer less than or equal to $a$). Prove that $\cdot$ is a binary operation on $G$ and that $G$ is a group.
   \begin{proof}Proof that $\cdot$ is a binary operator

   \end{proof}
   \begin{proof} Proof that G is a group
     \begin{enumerate}
       \item Proof of associativity over operator
       \item Proof of existence of identity element
       \item  Proof of existence of inverse of operator
     \end{enumerate}
   \end{proof}
%%%%%%%%%%%%%%%%%%%%%%%%%%%%%%%%%%%%%%%%%%%%%
\item Let $G = \{z \in \C \, : \, z^n = 1 \mbox{  for some nonnegative integer  } n\}$. Prove that $G$ is a group under multiplication (called the groups of {\it roots of unity} in $\C$).
\begin{proof}
  \begin{enumerate}
    \item Proof of associativity over operator
    \item Proof of existence of identity element
    \item  Proof of existence of inverse of operator
  \end{enumerate}
\end{proof}
%%%%%%%%%%%%%%%%%%%%%%%%%%%%%%%%%%%%%%%%%%%%%
\item Let $G = \{ a + b\sqrt{2} \, : \, a, b \in \Q\}$.
  \begin{itemize}
  \item[a)] Prove that $G$ is a group under addition.
  \begin{proof}
    \begin{enumerate}
      \item Proof of associativity over operator
      \item Proof of existence of identity element
      \item  Proof of existence of inverse of operator
    \end{enumerate}
  \end{proof}
  %%%%%%%%%%%%%%%%%%%%%%%%%%%%%%%%%%%%%%%%%%%%%
   \item[b)] Prove that the nonzero elements of $G$ are a group under multiplication [``Rationalize the denominators'' to find the inverses].
   \begin{proof}
     \begin{enumerate}
       \item Proof of associativity over operator
       \item Proof of existence of identity element
       \item  Proof of existence of inverse of operator
     \end{enumerate}
   \end{proof}
  \end{itemize}
  %%%%%%%%%%%%%%%%%%%%%%%%%%%%%%%%%%%%%%%%%%%%%
 \end{enumerate}



\end{document}
