\documentclass[11pt]{article}
\usepackage{amssymb,latexsym,amsmath,amsthm,graphicx, cite}
\usepackage{hyperref}
\hypersetup{
     colorlinks=true,
     linkcolor=blue,
     filecolor=blue,
     citecolor = black,      
     urlcolor=blue,
     }
%\usepackage[sc]{mathpazo}
%\linespread{1.05}         % Palatino needs more leading (space between lines)
%\usepackage[T1]{fontenc}
\usepackage{mathptmx}
\usepackage{multirow}
\usepackage{float}
\restylefloat{table}
\hoffset=0in 
\voffset=-.3in
\oddsidemargin=0in
\evensidemargin=0in
\topmargin=0in 
\textwidth=6.5in
\textheight=8.8in
\marginparwidth 0pt
\marginparsep 10pt
\headsep 10pt

\theoremstyle{definition}  % Heading is bold, text is roman
\newtheorem{theorem}{Theorem}
\newtheorem{corollary}{Corollary}
\newtheorem{definition}{Definition}
\newtheorem{example}{Example}
\newtheorem{proposition}{Proposition}



\newcommand{\Z}{\mathbb{Z}}
\newcommand{\N}{\mathbb{N}}
\newcommand{\Q}{\mathbb{Q}}
\newcommand{\R}{\mathbb{R}}
\newcommand{\C}{\mathbb{C}}

\newcommand{\lcm}{\mathrm{lcm}}



\setlength{\parskip}{0cm}
%\renewcommand{\thesection}{\Alph{section}}
\renewcommand{\thesubsection}{\arabic{subsection}}
\renewcommand{\thesubsubsection}{\arabic{subsection}.\arabic{subsubsection}}
\bibliographystyle{amsplain} 

%\input{../header}


\begin{document}

%\homework{}{Homework X}

\begin{enumerate}

\item Let $r$ represent the rigid motion of a regular $n$-gon ($n \geq 3$) which rotates the figure counterclockwise by $2\pi/n$. Let $s$ represent the rigid motion which reflects
  the figure along the axis of symmetry passing through the vertex labeled by $1$. 
  \begin{itemize}
  \item[a)] Carefully prove that $rs = sr^{-1}$ by computing the image of each vertex under the rigid motions
    $rs$ and $sr^{-1}$. Conclude that $D_{n}$ is a non-abelian group.
  \item[b)] Using induction prove that $r^ks = sr^{-k}$ for all positive integers $k$.
   \item[c)] We saw that $D_{n} = \{ e, r, r^2, \ldots, r^{n-1}, s, sr, sr^2, \ldots, sr^{n-1} \}$. Show that $|sr^k| = 2$ for all $k=0,1, \ldots,~{n-1}$.  
   \item[d)] Compute the index of the subgroup $\langle r \rangle$ in $D_{n}$. Do the same for $\langle s \rangle$. 
  \end{itemize}
\item Let $\Delta$ be a regular tetrahedron and let $G$ be the group of rigid motions of $\Delta$ [we are allowed to move $\Delta$ around in $\R^3$, but note
  that we cannot do reflections anymore ]. Show that $|G| = 12$. [Hint: consider the argument with which we showed $|D_{n}| = 2n$]
\item Determine all subgroups of $D_4$ and decide which ones are normal. Then find two subgroups $H_1$ and $H_2$ in $D_4$ such that $H_1 \subset H_2$,
  $H_1$ is normal in $H_2$, $H_2$ is normal in $D_4$ but $H_1$ is not normal in $D_4$. This show that ``is a normal subgroup of'' is not a transitive relation. 
\item Let $G$ be a group and $Z(G) = \{ g \in G \, : \, ga = ag \mbox{ for all } a \in G\}$. We have shown that $Z(G)$ is a subgroup of $G$. Now show that $Z(G)$
  is a normal subgroup of $G$. 
\item Let $H$ be a subgroup of $G$.
  \begin{itemize}
     \item[a)] For any $g \in G$, prove that $gHg^{-1}$ is a subgroup of $G$.
     \item[b)] Now suppose that $H$ is the unique subgroup of order $k$ in $G$. Prove that $H$ is a normal subgroup. 
   \end{itemize}
\end{enumerate}



\end{document}



