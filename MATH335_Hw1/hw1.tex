\documentclass[11pt]{article}
\usepackage{amssymb,latexsym,amsmath,amsthm,graphicx, cite}
\usepackage{hyperref}
\hypersetup{
     colorlinks=true,
     linkcolor=blue,
     filecolor=blue,
     citecolor = black,
     urlcolor=blue,
     }
%\usepackage[sc]{mathpazo}
%\linespread{1.05}         % Palatino needs more leading (space between lines)
%\usepackage[T1]{fontenc}
\usepackage{mathptmx}
\usepackage{multirow}
\usepackage{float}
\restylefloat{table}
\hoffset=0in
\voffset=-.3in
\oddsidemargin=0in
\evensidemargin=0in
\topmargin=0in
\textwidth=6.5in
\textheight=8.8in
\marginparwidth 0pt
\marginparsep 10pt
\headsep 10pt

\theoremstyle{definition}  % Heading is bold, text is roman
\newtheorem{theorem}{Theorem}
\newtheorem{corollary}{Corollary}
\newtheorem{definition}{Definition}
\newtheorem{example}{Example}
\newtheorem{proposition}{Proposition}



\newcommand{\Z}{\mathbb{Z}}
\newcommand{\N}{\mathbb{N}}
\newcommand{\Q}{\mathbb{Q}}
\newcommand{\R}{\mathbb{R}}
\newcommand{\C}{\mathbb{C}}

\newcommand{\lcm}{\mathrm{lcm}}



\setlength{\parskip}{0cm}
%\renewcommand{\thesection}{\Alph{section}}
\renewcommand{\thesubsection}{\arabic{subsection}}
\renewcommand{\thesubsubsection}{\arabic{subsection}.\arabic{subsubsection}}
\bibliographystyle{amsplain}

%\input{../header}


\begin{document}

%\homework{}{Homework I}

\begin{enumerate}

\item Let's start with a warm-up exercise for doing proofs where you need to use induction:  Prove that for all $n \in \N$
  $$ 1 + 2 + 2^2 + 2^3 + \cdots + 2^n \, = \, 2^{n+1} - 1.$$
  \textbf{Re-expressing the problem statement: }\\
  Prove that $\forall n \in \N$
  \[
    1+\sum_{i=1}^n2^i=2^{n+1}-1
  \]
  \begin{proof}\\
    Base case: n=2\\
    $$1+2^1+2^2=2^3-1$$
    $$7=7$$
    Inductive Hypothesis: \\
    $$P(k)=  1+\sum_{i=1}^k2^i=2^{k+1}-1$$
    \begin{align*}
      &P(k+1):\\\\
      &1+\sum_{i=1}^{k+1}2^i=2^{k+2}-1\\\\
      &1+\sum_{i=1}^{k}2^i+2^{k+1}=2^{k+2}-1\\\\
      &2^{k+1}-1+2^{k+1}=2^{k+2}-1\\\\
        &2(2^{k+1})-1=2^{k+2}-1\\\\
          &2^{k+2}-1=2^{k+2}-1\\\\
    \end{align*}
  \end{proof}
\item And here is a second warm-up exercise: if $x$ is a nonnegative real number show that $(1 + x)^n - 1 \geq nx$ for $n=0, 1, 2, \ldots$.\\
\textbf{Re-expressing the problem statment:}\\
let $x\in \R, x\geq 0$, Show that
\[
  (1 + x)^n - 1 \geq nx, \forall n \in \mathbb{W}
\]
\begin{proof}
  Suppose  $x\in \R, x\geq 0$
\begin{align*}
  &(1 + x)^n - 1 \geq nx\\
  &(1 + x)^n\geq nx+1\\
  &\sum_{k=0}^n\binom{n}{k}x^k\geq nx+1\\
  &nx+1+\sum_{k=2}^n\binom{n}{k}x^k\geq nx+1\\
  &\therefore (1 + x)^n\geq nx+1\\
  &\text{ Which implies }\\
  &(1 + x)^n-1\geq nx
\end{align*}
\end{proof}
\item Show that if $p$ is a prime number, there do not exist nonzero integers $a$ and $b$ such that $a^2 = pb^2$ (i.e. $\sqrt{p}$ is not a rational number).\\
\begin{Proof} If p is a prime number then the only divisors of p are 1 and p\\
Let a and be expressed in the prime factorization:
\[
  a=\prod_{i=1}^kP_i^{\alpha_i}
\]
\[
  b=\prod_{i=1}^kP_i^{\beta_i}
\]
the original statement is then:
\[
  \left[\prod_{i=1}^kp_i^{\alpha_i}\right]^2=p\left[\prod_{i=1}^kp_i^{\beta_i}\right]^2
\]
\[
  \prod_{i=1}^kp_i^{2\alpha_i}=p\prod_{i=1}^kp_i^{2\beta_i}\
\]
If $a^2=b^2$ then they would have identical prime factorizations. Since the integer is raised to the second power this implies that for all prime factors, each has an even exponent. \\
if p is not a prime factor of b then it would have an odd exponent. If p was a prime factor of b then it would form an odd exponent when combined with the corresponding prime factor since the each prime factor in $b^2$ has an even exponent.  In either case p exists as a prime number with an odd exponent therefore it cannot be equal to a since a has a prime factorization that has eveen exponenets for each prime factor.
\end{Proof}
\item Compute the $\gcd$ of the following pairs of integers:
  \begin{itemize}
  \item[i)] $14$ and $39$;
  \begin{align*}
    &39=14(2)+11\\
    &14=11(1)+3\\
    &11=3(3)+2\\
    &3=2(1)+1\\
    &2=1(2)+0\\
  \end{align*}
    gcd(39,14)=1
  \item[ii)] $234$ and $165$;
  \begin{align*}
  &234=165(1)+69\\
  &165=69(2)+27\\
  &69=27(2)+15\\
  &27=15(1)+12\\
  &15=12(1)+3\\
  &12=3(4)+0\\
\end{align*}
  gcd(234,165)=3
  \item[iii)] $471$ and $562$.
  \begin{align*}
  &562=471(1)+91\\
  &471=91(5)+16\\
  &91=16(5)+11\\
  &16=11(1)+5\\
  &11=5(2)+1\\
  &5=1(5)+0\\
\end{align*}
gcd(562,471)=1
  \end{itemize}
\item Let $a, b, c \in \Z$. Prove that if $\gcd(a,b) =1$ and $a \, | \, bc$ then $a | c$.
\begin{proof}
if $a|bc$ then $bc=ak,k \in \Z$
\end{proof}

\item Let $a$ and $b$ be positive integers where
  $$ a = p_1^{\alpha_1} p_2^{\alpha_2} \cdots p_k^{\alpha_k}  \quad \quad \mbox{and}  \quad \quad a = p_1^{\beta_1} p_2^{\beta_2} \cdots p_k^{\beta_k}  $$
  with $p_1, \ldots, p_k$ distinct primes and $\alpha_i, \beta_i \geq 0$ for $i=1, \ldots, k$. Show that
  $$ \lcm(a,b) \, = \, p_1^{\max(\alpha_1, \beta_1)} p_2^{\max(\alpha_2, \beta_2)} \cdots p_k^{\max(\alpha_k,\beta_k)}. $$


\begin{proof}
  Let a and b be expressed in a product form as follows:
  \[
    a=\prod_{i=1}^kp_i^{\alpha_i}
  \]
  \[
    b=\prod_{i=1}^kp_i^{\beta_i}
  \]
  Let:
  \[
    l=\prod_{i=0}^kp_i^{\max(\alpha_i,\beta_i)}
  \]
  be the least common multiple of a and b.

  \begin{enumerate}
    \item a
  \end{enumerate}
\end{proof}
 \end{enumerate}



\end{document}
