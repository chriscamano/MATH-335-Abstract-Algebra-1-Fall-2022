\documentclass[11pt]{article}
\usepackage{amssymb,latexsym,amsmath,amsthm,graphicx, cite}
\author{Chris Camano: ccamano@sfsu.edu}
\title{MATH 335  Lecture 24 }
\date

\usepackage[many]{tcolorbox}
\tcbset{breakable}
\usepackage{mathptmx}
\usepackage{multirow}
\usepackage{float}
\restylefloat{table}
\hoffset=0in
\voffset=-.3in
\oddsidemargin=0in
\evensidemargin=0in
\topmargin=0in
\textwidth=6.5in
\textheight=8.8in
\marginparwidth 0pt
\marginparsep 10pt
\headsep 10pt

\theoremstyle{definition}  % Heading is bold, text is roman
\newtheorem{theorem}{Theorem}
\newtheorem{corollary}{Corollary}
\newtheorem{defn}{Definition}
\newtheorem{example}{Example}
\newtheorem{proposition}{Proposition}



\newcommand{\Z}{\mathbb{Z}}
\newcommand{\N}{\mathbb{N}}
\newcommand{\Q}{\mathbb{Q}}
\newcommand{\R}{\mathbb{R}}
\newcommand{\C}{\mathbb{C}}
\newcommand{\lcm}{\mathrm{lcm}}
\setlength{\parskip}{0cm}
%\renewcommand{\thesection}{\Alph{section}}
\renewcommand{\thesubsection}{\arabic{subsection}}
\renewcommand{\thesubsubsection}{\arabic{subsection}.\arabic{subsubsection}}
\bibliographystyle{amsplain}

%\input{../header}
\newcommand{\bigline}{\\\noindent\makebox[\linewidth]{\rule{\textwidth}{0.4pt}}\\}

\newcommand{\block}[2]{\begin{tcolorbox}[title={#1}]{#2}\end{tcolorbox}}
\begin{document}
\maketitle
\block{Definition}{
A subgroup H of a group G is called a normal subgroup if $gH=Hg\quad \forall\g\in G$\\\\
Equivilantly H is a normal subgroup if and only if: $gHg^{-1}=H$ for all g in G.
}
\\
\textbf{Example:}\\
Every subgroup of an abelian group is normal. \\
\textbf{Example}\\
Let $G=S_3$ and $H=<(1 2)>=\{e,(1 2)\}$ We showed last time that H is not normal . In general we can prove this by showing tha $gHg^{-1}$ is not equal to H. \\
All subgroups of index two must automatically be a normal subgroup. \\
Homework hint: Given a group G consider the following set : $Z(G)=\{g\in G:ga=ag\forall a\in G\}$ Show that Z(G) is a normal subgroup of G. This is always true the abelian subgroup of a group is always a normal subgroup of G.\\
\block{Motivation: Factor Groups}{
Let G be a group and H is a subgroup of G. We consider all distinct left cosets of H in G.
The set formed by all left cosets of G is denoted:
$G/H$ We wish to make this set of all distinct left cosets of H in G into a group\\
Intuitivley it feels like this can be accomplished by the following let a and b the representatives of two left cosets in H then::
\[
  (aH)(bH)=abH
\]
is our binary operation. but this is not well defined because selection of different representatives leads to different outcomes meaning that it fails as the binary operation. If H is a normal subgroup then this binary operation is sufficient
}
\block{Definition}{
Let G be a gorup and N a norma subgroup of G\\
If $aN=a^*N$ and $bN=b^*N$ then $abN=a^*b^*N$

}
\begin{proof}
  Let $a^*=an$ for some n in N and $b^*=bn$ for some other n in N. Then we have
  \[
    a^*b^*N=an_1bn_2N=an_1bN=an_1Nb=aNb=abN
  \]
  We are able to convert a left coset to a right coset since we know that N is a norla sugroup giving us the ability to commuete right and left subgroups as we please throughout our proofs.
\end{proof}
\block{Theorem:}{
If N is a normalsubgroup of the group G then the binary operation defined by $(aN)(bN)=(ab)N$ makes G/N the set of distinct left cosets
}

To prove this statement we must prove the three propeties that form a group  all three can be proven using the extended properties of the original group. This group is called a factor group or quotient group


\\\\
If G is finite and N is a normal subgroup then the factor group: $G/N$ is the index of G given N. \[
  \left|G/N\right |=\frac{|G|}{|N|}
\]

\end{document}
