\documentclass[11pt]{article}
\usepackage{amssymb,latexsym,amsmath,amsthm,graphicx, cite}
 \author{Chris Camano: ccamano@sfsu.edu}
 \title{MATH 335  lecture 15 }
 \date

\usepackage{mathptmx}
\usepackage{multirow}
\usepackage{float}
\restylefloat{table}
\hoffset=0in
\voffset=-.3in
\oddsidemargin=0in
\evensidemargin=0in
\topmargin=0in
\textwidth=6.5in
\textheight=8.8in
\marginparwidth 0pt
\marginparsep 10pt
\headsep 10pt

\theoremstyle{definition}  % Heading is bold, text is roman
\newtheorem{theorem}{Theorem}
\newtheorem{corollary}{Corollary}
\newtheorem{defn}{Definition}
\newtheorem{example}{Example}
\newtheorem{proposition}{Proposition}



\newcommand{\Z}{\mathbb{Z}}
\newcommand{\N}{\mathbb{N}}
\newcommand{\Q}{\mathbb{Q}}
\newcommand{\R}{\mathbb{R}}
\newcommand{\C}{\mathbb{C}}

\newcommand{\lcm}{\mathrm{lcm}}



\setlength{\parskip}{0cm}
%\renewcommand{\thesection}{\Alph{section}}
\renewcommand{\thesubsection}{\arabic{subsection}}
\renewcommand{\thesubsubsection}{\arabic{subsection}.\ar  abic{subsubsection}}
\bibliographystyle{amsplain}

%\input{../header}

\newcommand{\bigline}{\\\noindent\makebox[\linewidth]{\rule{\paperwidth}{0.4pt}}\\}

\begin{document}
\maketitle
\defn Let G be a group and pick any element$a\in G$ then:
\[
  <a>=\{a^k, k \in \Z\}
\]
Where negative exponents correlate with composition over the group operator with the inverse. This sub group is called the cyclic subgroup generated by a. A cyclic subgroup always has at least two generators since $<a>=<a^{-1}>$\\\\
It could be the case as well that there exists a generator that generates the entire group to begin with: \\
If \[
  G=<a>
\]
Then we say G is a cyclic group. \\
\proposition $\Z$ is a cyclic group since $\Z=<1>$. In general for the group $\Z$ $a^k=ka$ since addition consectuive times is the same as multiplication over n times. This cyclic group is infinite. When a cyclic group is inifinite it is always $\Z$\\\\
For each positive integer n $\Z_n$ is also a cyclic group since we have the property that under a modular operator we observe cyclic behavior as the equivilance classes loop back to the identity at every multiple of n. this is to say:
\[
  \Z_n=<[1]>
\]
\\\\
Not every group is cyclic, for eexample all symmetry groups are not cyclic. \\\\
\theorem Every cyclic group is abelian\\
\begin{proof }
  Let G be a cyclic group, this is to say that G is generated by one element in G , g:\\
  let $a,b\in G$ these two elements are some powers of the generator g meaning:
  \[
    a=g^k\quad b=g^l
  \]
  \[
    ab=g^lg^l=g^{k+l}=g^lg^k=ba
  \]

\end{proof }
\theorem Every subgroup of a cyclic group is cyclic
\begin{proof}
  Let G be a cyclic group so $G=<g>$ let $H\leq G$ If H=$\{e}$ then H is cyclic because H=$<e>$\\
  Let $H\neq \{e}$Then for some positive integer k, $g^k\in H$ since all elements of G are of the form $g^k$Suppose $g^l\in H, but l<0.$ but l$<0$ so $g^{-l}\in H$\\
  Let d be the smallest postive integer so that $g^d\in H$ by the well ordering property. \\\\
  \textbf{Claim:}$H=<g^d>$\\
  We first show that $<g^d>\subset H$:\\
  \[
    <g^d>=\{g^{dk},k\in \Z\}
  \]
  $g^d$ is in H and H is a subgroup which means that compositions over $g^d$ composed with itself is an element of H by the closure of the group operator.\\
  We now show that $H\subset <g^d>$\\
  let $a\in H$ so we know that $a=g^n,n\in Z$
  We then need to show that d divides n which is akin to proving that there is no remainder when we divide n by d. by division algorithm show:
  \[
    n=qd+r \quad 0\leq r<d
  \]
  if r =0 we are done, so lets suppose r<0, then:
  suppose $g^n\in H$ then with our realtion we have:
  \[
    g^n=g^{qn+r}=g^{d^q}g^r
  \]
  \[
    g^{d^q}\in H
  \]
  by definition of H since we have $g^r\in H$ and $g^{d^q}\in H $ we know that $g^{d^q}g^r\in H$ since we are composing two elements that are both in H. r is assumed to be positive but is less than d, but d is the smallest integer meaning this cannot be the case.
\end{proof}
\defn All subgroups of $\Z$ is of the form:
\[
  <n>=n\Z=\{kn:k\in \Z\}

\]
\defn Let G be a group and $a\in G$ then the smalles positive integer n such that: $a^n=e$ is called the order of a denoted as $|a|$\\
\proposition Let G be a group, $a\in G$ such taht $|a|=n$:\\
Then \[
  <a>=\{e,a,a^2,...,a^{n-1}\}
\]

\end{document}
