\documentclass[11pt]{article}
\usepackage{amssymb,latexsym,amsmath,amsthm,graphicx, cite}
\author{Chris Camano: ccamano@sfsu.edu}
\title{MATH 335  Lecture 25 }
\date

\usepackage[many]{tcolorbox}
\tcbset{breakable}
\usepackage{mathptmx}
\usepackage{multirow}
\usepackage{float}
\restylefloat{table}
\hoffset=0in
\voffset=-.3in
\oddsidemargin=0in
\evensidemargin=0in
\topmargin=0in
\textwidth=6.5in
\textheight=8.8in
\marginparwidth 0pt
\marginparsep 10pt
\headsep 10pt

\theoremstyle{definition}  % Heading is bold, text is roman
\newtheorem{theorem}{Theorem}
\newtheorem{corollary}{Corollary}
\newtheorem{defn}{Definition}
\newtheorem{example}{Example}
\newtheorem{proposition}{Proposition}



\newcommand{\Z}{\mathbb{Z}}
\newcommand{\N}{\mathbb{N}}
\newcommand{\Q}{\mathbb{Q}}
\newcommand{\R}{\mathbb{R}}
\newcommand{\C}{\mathbb{C}}
\newcommand{\lcm}{\mathrm{lcm}}
\setlength{\parskip}{0cm}
%\renewcommand{\thesection}{\Alph{section}}
\renewcommand{\thesubsection}{\arabic{subsection}}
\renewcommand{\thesubsubsection}{\arabic{subsection}.\arabic{subsubsection}}
\bibliographystyle{amsplain}

%\input{../header}
\newcommand{\bigline}{\\\noindent\makebox[\linewidth]{\rule{\textwidth}{0.4pt}}\\}

\newcommand{\block}[2]{\begin{tcolorbox}[title={#1}]{#2}\end{tcolorbox}}
\begin{document}
\maketitle
\block{An example on Factor Groups}{Let $G=
\Z$Since this group is abelian ever subgroup of G is normal. Any subgroup of $\Z$ is of the form $<n>=n\Z$\\\\
We not consider the factor group $\Z/n\Z$ which is the set of all left cosets of $n\Z$ in $\Z$\\\\
The left cosets are of the form $m+n\Z$,
\[
  n\Z=\{kn:k\in \Z\}\quad \Z/n\Z=\{m+kn:k\in \Z\}
\]
\[
  \Z/n\Z=\{n\Z,1+n\Z,2+n\Z,...,(n-1)+n\Z\}
\]
The factor group is a group under the operation:
\[
  (a+n\Z)+(b+n\Z)=(a+b\mod n)+n\Z
\]
}
The groups $\Z/n\Z and \Z_n$ are isomorphic to one another since their elements are equivilant, but their construction is different.


\block{homomorphism}{
A homomorphism between the groups $(G_1,*),(G_2,\circ)$ is a map (function) from the first group to the second one. Let $$\phi:G_1\mapsto G_2$$, such that $$a,b\in G_1 $$  $$\phi(a*b)=\phi(a)\circ \phi(b)$$


}
Proving that a group homomorphism is bijective is equivilant to proving that two groups are isomorphic!!!!!!!!!


\block{Proposition}{

Let $\phi : G_1\mapsto G_2$ be a group homomorphism . Then :
\begin{enumerate}
  \item
  \[
    \phi(e_{G_1})=e_{G_2}
  \]
  \begin{proof}
    $\newline$
    $\phi(e_{G_1})=\phi(e_{G_1}e_{G_1})=\phi(e_{G_1})\phi(e_{G_2})$
    $e_{G_2}=\phi(e_{G_1})$
  \end{proof}
  \item
  $\forall g\in G $
  \[
    \phi(g^{-1})=\phi(g)^{-1}
  \]
  \item
  \begin{proof}
    $\newline$
    \[
      \phi(e_{G_1})=\phi(gg^{-1})=\phi(g)\phi(g^{-1})=e_{G_2}
    \]

  \end{proof
  \item Let $H_1$ be a subgroup of $G_1$, then \[
    \phi(H_1)=\{\phi(h):h\in H_1\}
  \]
  is a subgroup of $G_2$
\end{enumerate}

}
\end{document}
