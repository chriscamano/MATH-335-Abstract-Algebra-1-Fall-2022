\documentclass[11pt]{article}
\usepackage{amssymb,latexsym,amsmath,amsthm,graphicx, cite}
\author{Chris Camano: ccamano@sfsu.edu}
\title{MATH 335  Lecture 22 }
\date

\usepackage[many]{tcolorbox}
\tcbset{breakable}
\usepackage{mathptmx}
\usepackage{multirow}
\usepackage{float}
\restylefloat{table}
\hoffset=0in
\voffset=-.3in
\oddsidemargin=0in
\evensidemargin=0in
\topmargin=0in
\textwidth=6.5in
\textheight=8.8in
\marginparwidth 0pt
\marginparsep 10pt
\headsep 10pt

\theoremstyle{definition}  % Heading is bold, text is roman
\newtheorem{theorem}{Theorem}
\newtheorem{corollary}{Corollary}
\newtheorem{defn}{Definition}
\newtheorem{example}{Example}
\newtheorem{proposition}{Proposition}



\newcommand{\Z}{\mathbb{Z}}
\newcommand{\N}{\mathbb{N}}
\newcommand{\Q}{\mathbb{Q}}
\newcommand{\R}{\mathbb{R}}
\newcommand{\C}{\mathbb{C}}
\newcommand{\lcm}{\mathrm{lcm}}
\setlength{\parskip}{0cm}
%\renewcommand{\thesection}{\Alph{section}}
\renewcommand{\thesubsection}{\arabic{subsection}}
\renewcommand{\thesubsubsection}{\arabic{subsection}.\arabic{subsubsection}}
\bibliographystyle{amsplain}

%\input{../header}
\newcommand{\bigline}{\\\noindent\makebox[\linewidth]{\rule{\textwidth}{0.4pt}}\\}

\newcommand{\block}[2]{\begin{tcolorbox}[title={#1}]{#2}\end{tcolorbox}}
\begin{document}
\maketitle
\block{Definition: The Dihedral Group}{
The dihedral Group is a concet that emerges from the study of a regular convex n-gon, n$\geq$ 3 in the cartesian plane. convexity in this context meaning the edges are all pointing outwards regularity indicating uniform edge length between vertices.
}
Typically the dihedral roup is denoted as: $D_{2n}$ where n is the degree of the n gon. \\
The elements of $D_{2n}$ are the rigid motions of the n gon.
\block{Definition: Rigid motion}{
A rigid motion is a motion that enables a shape to be placed back in place after transformation.
}
\block{Definition:$D_6$=\{$ S_3$\}=\{r,s,$r\circ s$, $s\circ r $,e,$r^2$\}}
{}
\block{proposition: $|D_{2n}|=2n$}{
}
\begin{proof}
  Under a ridig motion of an n gon vertex 1 has to be placed at one of the n vertices and vertex 2 has 2 possible places to occupy.
\end{proof}
\block{Definition=$<r>\subset D_{2n}$}
{We know this since $|<r>|=n and |<r>\circ s|\circ s>|=n$}



\[
  D_{2n}=<r>+s\circ<r>
\]
\[
  r^ks=sr^{-k}
\]
\[
  D_{2n}=s\circ <r>\cup<r>
\]
\end{document}
