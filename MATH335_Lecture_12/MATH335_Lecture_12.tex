\documentclass[11pt]{article}
\usepackage{amssymb,latexsym,amsmath,amsthm,graphicx, cite}
 \author{Chris Camano: ccamano@sfsu.edu}
 \title{MATH 335  lecture 12 }
 \date

\usepackage{mathptmx}
\usepackage{multirow}
\usepackage{float}
\restylefloat{table}
\hoffset=0in
\voffset=-.3in
\oddsidemargin=0in
\evensidemargin=0in
\topmargin=0in
\textwidth=6.5in
\textheight=8.8in
\marginparwidth 0pt
\marginparsep 10pt
\headsep 10pt

\theoremstyle{definition}  % Heading is bold, text is roman
\newtheorem{theorem}{Theorem}
\newtheorem{corollary}{Corollary}
\newtheorem{definition}{Definition}
\newtheorem{example}{Example}
\newtheorem{proposition}{Proposition}



\newcommand{\Z}{\mathbb{Z}}
\newcommand{\N}{\mathbb{N}}
\newcommand{\Q}{\mathbb{Q}}
\newcommand{\R}{\mathbb{R}}
\newcommand{\C}{\mathbb{C}}

\newcommand{\lcm}{\mathrm{lcm}}



\setlength{\parskip}{0cm}
%\renewcommand{\thesection}{\Alph{section}}
\renewcommand{\thesubsection}{\arabic{subsection}}
\renewcommand{\thesubsubsection}{\arabic{subsection}.\ar  abic{subsubsection}}
\bibliographystyle{amsplain}

%\input{../header}


\begin{document}
\maketitle
\section{Opening Notes}
The final exam will only be an inclass exam without a take home component Hosten will be out of the office during the final so we will only be doing an in class exam. This means that we will have two hours to complete a test. The difficulty will be somewhere between the regular in class exam and the take home exam . \\
The final exam will not be curved since the general grading system is generous to begin with.
\section{New Material}
\[
  S_3=\{\textbf{The set of biections of \{1,2,3}\}\}
\]
$$S_3=\left\{
\begin{bmatrix}
  1&2&3\\
  1&2&3
\end{bmatrix},
\begin{bmatrix}
  1&2&3\\
  1&3&2
\end{bmatrix},
\begin{bmatrix}
  1&2&3\\
  3&2&1
\end{bmatrix},
\begin{bmatrix}
  1&2&3\\
  2&1&3
\end{bmatrix},
\begin{bmatrix}
  1&2&3\\
  2&3&1
\end{bmatrix},
\begin{bmatrix}
  1&2&3\\
  3&1&2
\end{bmatrix}\right\}$$
Operation on $S_3$: Composition of bijections.Under composition it is not abelian.
Verify this is a group
\begin{enumerate}
  \item Closure under operator\\
  \item Associativity\\
  In general function composition is associative therefore:
  \[
    \forall \sigma, \mu, \tau \in S_3: \sigma \circ(\tau \circ \mu)=(\sigma \circ \tau)\circ \mu
  \]
  \item Identity element \\
  \[
  \begin{bmatrix}
    1&2&3\\
    1&2&3
  \end{bmatrix},
  \]
  \item Inverse\\
  for the fixed point functions you can compose with itself to get back to identity. For rotations compose with opposite rotation. There exists an inverse for all bijections, since S3 is the set of all bijections it is implied that the inverse is an element of $S_3$
\end{enumerate}
\theorem: $S_3$ is a non abelian finite cyclcic group under function composition.
\theorem: Let $S_n$ be the set of bijections of$\{1,2,3,4,...,n\}$ Then $S_n$ is a non abelian group under function composition.
\theorem $|S_n|$ is n!, Since this the number of permutations for a set of n elements.
\proposition $S_n$ is the symmetric group of order n. All finite sets of bijections are subsets of $S_n$.
\end{document}
