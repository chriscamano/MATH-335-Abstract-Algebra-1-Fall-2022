\documentclass[11pt]{article}
\usepackage{amssymb,latexsym,amsmath,amsthm,graphicx, cite}
\usepackage{hyperref}
\hypersetup{
     colorlinks=true,
     linkcolor=blue,
     filecolor=blue,
     citecolor = black,      
     urlcolor=blue,
     }
%\usepackage[sc]{mathpazo}
%\linespread{1.05}         % Palatino needs more leading (space between lines)
%\usepackage[T1]{fontenc}
\usepackage{mathptmx}
\usepackage{multirow}
\usepackage{float}
\restylefloat{table}
\hoffset=0in 
\voffset=-.3in
\oddsidemargin=0in
\evensidemargin=0in
\topmargin=0in 
\textwidth=6.5in
\textheight=8.8in
\marginparwidth 0pt
\marginparsep 10pt
\headsep 10pt

\theoremstyle{definition}  % Heading is bold, text is roman
\newtheorem{theorem}{Theorem}
\newtheorem{corollary}{Corollary}
\newtheorem{definition}{Definition}
\newtheorem{example}{Example}
\newtheorem{proposition}{Proposition}



\newcommand{\Z}{\mathbb{Z}}
\newcommand{\N}{\mathbb{N}}
\newcommand{\Q}{\mathbb{Q}}
\newcommand{\R}{\mathbb{R}}
\newcommand{\C}{\mathbb{C}}

\newcommand{\lcm}{\mathrm{lcm}}



\setlength{\parskip}{0cm}
%\renewcommand{\thesection}{\Alph{section}}
\renewcommand{\thesubsection}{\arabic{subsection}}
\renewcommand{\thesubsubsection}{\arabic{subsection}.\arabic{subsubsection}}
\bibliographystyle{amsplain} 

%\input{../header}


\begin{document}

%\homework{}{Homework II}

\begin{enumerate}

\item Produce a clear and clean proof of the following statement you have discovered in class:  Let $a$ and $b$ two positive integers. Then
$ab = \gcd(a,b) \lcm(a,b)$. 
\item We learned that if two integers $a$ and $b$ are relatively prime, then there exist integers $t$ and $u$ such that $at + bu = 1$.
Prove the converse: if there are integers $t$ and $u$ such that $at + bu = 1$ then $a$ and $b$ are relatively prime. 
\item Let $a$ and $b$ be integers with $b >0$. Using division algorithm, write $a = bq + r$ where $q, r \in \Z$ and $0 \leq r < b$. Show that $\gcd(a,b) = \gcd(b, r)$.
\item Show that for all positive integers $n > 2$, $\phi(n)$ is an even number. 
\item Prove that if $d$ divides $n$ then $\phi(d)$ divides $\phi(n)$. 


 \end{enumerate}



\end{document}



