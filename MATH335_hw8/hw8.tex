\documentclass[11pt]{article}
\usepackage{amssymb,latexsym,amsmath,amsthm,graphicx, cite}
\usepackage{hyperref}
\hypersetup{
     colorlinks=true,
     linkcolor=blue,
     filecolor=blue,
     citecolor = black,      
     urlcolor=blue,
     }
%\usepackage[sc]{mathpazo}
%\linespread{1.05}         % Palatino needs more leading (space between lines)
%\usepackage[T1]{fontenc}
\usepackage{mathptmx}
\usepackage{multirow}
\usepackage{float}
\restylefloat{table}
\hoffset=0in 
\voffset=-.3in
\oddsidemargin=0in
\evensidemargin=0in
\topmargin=0in 
\textwidth=6.5in
\textheight=8.8in
\marginparwidth 0pt
\marginparsep 10pt
\headsep 10pt

\theoremstyle{definition}  % Heading is bold, text is roman
\newtheorem{theorem}{Theorem}
\newtheorem{corollary}{Corollary}
\newtheorem{definition}{Definition}
\newtheorem{example}{Example}
\newtheorem{proposition}{Proposition}



\newcommand{\Z}{\mathbb{Z}}
\newcommand{\N}{\mathbb{N}}
\newcommand{\Q}{\mathbb{Q}}
\newcommand{\R}{\mathbb{R}}
\newcommand{\C}{\mathbb{C}}

\newcommand{\lcm}{\mathrm{lcm}}



\setlength{\parskip}{0cm}
%\renewcommand{\thesection}{\Alph{section}}
\renewcommand{\thesubsection}{\arabic{subsection}}
\renewcommand{\thesubsubsection}{\arabic{subsection}.\arabic{subsubsection}}
\bibliographystyle{amsplain} 

%\input{../header}


\begin{document}

%\homework{}{Homework VIII}

\begin{enumerate}

\item Let $G$ be a group of order $60$. What are the possible orders of the subgroups of $G$ ?
\item Let $G$ be a group and let $a$ and $b$ be elements in $G$ with $|a| = n$ and $|b| = m$. If $gcd(m,n) = 1$, prove that $|G| \geq mn$. 
\item Let $G$ be a group and $H$ a subgroup of $G$. We proved that $g_1H = g_2H$ if and only if $g_2 \in g_1H$. Using similar ideas in the
  proof of this result prove that $g_1H=g_2H$ if and only if $g_1g_2^{-1} \in H$. 
\item First notation: Suppose $G$ is group and $H$ is a subgroup of $G$. We call the number of distinct left cosets of $H$ in $G$ the index of $H$ in $G$,
  and we denote this number by $[G \, : \, H]$. For instance, if $G$ is a finite group, the Lagrange's Theorem can be stated as $[G \, : \, H] = |G| / |H|$.
  Now prove that $[G \, : \, H]$ is also the number of distinct right cosets of $H$ in $G$.
 \item Show that if $[G \, : \, H] = 2$ then $gH = Hg$ for all $g \in G$. 
\end{enumerate}



\end{document}



