\documentclass[11pt]{article}
\usepackage{amssymb,latexsym,amsmath,amsthm,graphicx, cite}
\author{Chris Camano: ccamano@sfsu.edu}
\title{MATH 335  Homework 8? }
\date

\usepackage[many]{tcolorbox}
\tcbset{breakable}
\usepackage{mathptmx}
\usepackage{multirow}
\usepackage{float}
\restylefloat{table}
\hoffset=0in
\voffset=-.3in
\oddsidemargin=0in
\evensidemargin=0in
\topmargin=0in
\textwidth=6.5in
\textheight=8.8in
\marginparwidth 0pt
\marginparsep 10pt
\headsep 10pt

\theoremstyle{definition}  % Heading is bold, text is roman
\newtheorem{theorem}{Theorem}
\newtheorem{corollary}{Corollary}
\newtheorem{defn}{Definition}
\newtheorem{example}{Example}
\newtheorem{proposition}{Proposition}
\usepackage{listings}
\usepackage{xcolor}
\lstset { %
    language=C++,
    backgroundcolor=\color{black!5}, % set backgroundcolor
    basicstyle=\footnotesize,% basic font setting
}


\newcommand{\Z}{\mathbb{Z}}
\newcommand{\N}{\mathbb{N}}
\newcommand{\Q}{\mathbb{Q}}
\newcommand{\R}{\mathbb{R}}
\newcommand{\C}{\mathbb{C}}
\newcommand{\lcm}{\mathrm{lcm}}
\setlength{\parskip}{0cm}
%\renewcommand{\thesection}{\Alph{section}}
\renewcommand{\thesubsection}{\arabic{subsection}}
\renewcommand{\thesubsubsection}{\arabic{subsection}.\arabic{subsubsection}}
\bibliographystyle{amsplain}

%\input{../header}
\newcommand{\bigline}{\\\noindent\makebox[\linewidth]{\rule{\textwidth}{0.4pt}}\\}

\newcommand{\block}[2]{\begin{tcolorbox}[title={#1}]{#2}\end{tcolorbox}}
\begin{document}
\maketitle
%%%%%%%%%%%%%%%%%%%%%%%%%%%%%%%%%%%%%%%%%%%%%%%%%%%%%%%%%%%%%%
 \block{Question #1}{Let $G$ be a group of order $60$. What are the possible orders of the subgroups of $G$ ?}
 \begin{proof}
   For this proof we will invoke Lagranges theorem which states that for finite groups the order of any of its subgroups divide the order of the original group.\\\\
   The divisors of 60 are:,1,2,3,5,6,10,12,15,20,30,60
   Thus the possible subgroup orders are the values above with 1 being identity and 60 being G.
 \end{proof}
 %%%%%%%%%%%%%%%%%%%%%%%%%%%%%%%%%%%%%%%%%%%%%%%%%%%%%%%%%%%%%%
 \block{Question #2}{ Let $G$ be a group and let $a$ and $b$ be elements in $G$ with $|a| = n$ and $|b| = m$. If $gcd(m,n) = 1$, prove that $|G| \geq mn$. }
 \begin{proof}
   After much deliberation I have concluded that based onn the nature of this problem we can assume that G is finite. Why do I feel this assumption is valid? I think that this problem is testing our comprehension of Lagranges theorem which only holds for finite groups. For this proof I will proceed with G being finite \\
   Let $|G|=g$ then, since we have a cyclic subgroup generated by a with order n and a cyclic subgroup generated by b with order m, by Lagranges theorem:
   \[
     n|g\mapsto g=nu, u \in \Z \quad m|g\mapsto g=mv, v\in \Z
   \]
   Since n and m are coprime by bezout's identity we can write the following :
   \[
     ns+mt=1, \quad s,t in \Z
   \]
   Multiplying by g gives us:
   \begin{align*}
     & gns+gmt=g\\
     &mvns+numt=g\\
     &mn(vs+ut)=g\\
     &mn|g\\
     &g=mnl, l \in \N
   \end{align*}
   Here we see that mn divides the order of the group, and that there is a potential case of equivilancy for when the integer scalar on mn is equal to 1. Otherwise g is greater than mn and we arrive to our conclusion: \[
     |G|\geq mn
   \]
 \end{proof}
 %%%%%%%%%%%%%%%%%%%%%%%%%%%%%%%%%%%%%%%%%%%%%%%%%%%%%%%%%%%%%%
 \block{Question #3}{ Let $G$ be a group and $H$ a subgroup of $G$. We proved that $g_1H = g_2H$ if and only if $g_2 \in g_1H$. \\\\Using similar ideas in the
  proof of this result prove that $g_1H = g_2H$ if and only if $g_1^{-1}g_2 \in H$}
  \begin{proof}
    $\newline$
    \begin{enumerate}
      \item \textbf{Part 1 $(g_1H=g_2H )\rightarrow( g_1^{-1}g_2\in H) $}\\\\
      Suppose that $g_1H=g_2H$ then this implies $g_1h_1=g_2h_2$ for some $h_1,h_2\in H$
      \begin{align*}
        &g_1h_1=g_2h_2\\
        &g_1^{-1}(g_1h_1)=g_1^{-1}g_2h_2\\
        &h_1h_2^{-1}=g_1^{-1}g_2
      \end{align*}
      H is a subgroup so $h_1h_2^{-1} \in H$ which shows that $g_1^{-1}g_2\in H$ by equivilancy \\\\
      %%%%%%%%%%%%%%%%%%%%%%%%%%%%%%%%%%%%%%%
      \item \textbf{Part 2 $( g_1^{-1}g_2\in H) \rightarrow (g_1H=g_2H ) $ }\\\\
      Assume $g_1^{-1} g_2 \in H$ . If $g_1^{-1} g_2 \in H$ then we also know that the inverse $g_2^{-1} g_1 \in H$ since H is a subgroup. Given some arbitray element $h\in H$
      we can now make the following argument:
      \begin{align*}
        h=(g_1^{-1} g_2)(g_2^{-1} g_1)h
      \end{align*}
      $(g_2^{-1} g_1)h\in H$ as well so lets call it $\bar{h}$. then we arrive at the following:
      \begin{align*}
        &h=(g_1^{-1} g_2)\bar{h}\\
        &g_1h=g_2\bar{h}
      \end{align*}
      $g_1h\in g_1H$ and $g_2\bar{h}\in g_2H$ but here our work demonstrates that for an arbitrary element of $g_2H$ that it is also an element of $g_1H$ this holds for any selection of h and $\bar{h}$ meaning we have shown $g_1H=g_2H$ \\

      \\\\\\\\
    %  Assume $g_1^{-1}g_2\in H$ this is to say $g_1^{-%1}g_2=\bar{h}, \bar{h}\in H$\\
    %  This implies then that:
    %  \[
      %  g_2=\bar{h}g_1\quad g_1=\bar{h}^{-1}g_2
      %\]
      %Recall that:
      %\[
      %  g_2H=\{g_2h:h\in H\}=\{\bar{h}g_1h:h\in %H\}\subset g_1H
      %\]
      %Since $g_1=g_2\bar{h}^{-1}$
    %  \[
      %  g_1H=\{g_1h:h\in H\}=\{\bar{h}^{-1}g_2h:h\in %H\}\subset g_2H
    %  \]
    %  thus \[
      %  g_1H=g_2H
      %\]
    \end{enumerate}
  \end{proof}
 \block{Question #4}{ First notation: Suppose $G$ is group and $H$ is a subgroup of $G$. We call the number of distinct left cosets of $H$ in $G$ the index of $H$ in $G$,
  and we denote this number by $[G \, : \, H]$. \\\\For instance, if $G$ is a finite group, the Lagrange's Theorem can be stated as $[G \, : \, H] = |G| / |H|$.
  \\\\Now prove that $[G \, : \, H]$ is also the number of distinct right cosets of $H$ in $G$.}
  \begin{proof}
    \textbf{Lemma: There exists a bijection from a subgroup H of a group G and the right coset Hg for some $g\in G$}
    \begin{proof}
      Let$\phi:H\mapsto Hg$
      \[
        \phi(h)=hg\quad \forall h\in H
      \]
    \end{proof}
    \begin{enumerate}
      \item \textbf{Injectivity}\\
      Suppose $\phi(h_1)=\phi(h_2)$, Then:
      \[
        h_1g=h_2g\mapsto h_1=h_2
      \]
      So we have demonstrated injectivity through right multiplcication with g inverse.
      \item \textbf{Surjectivity}
      Any element in Hg is of the form $hg$ for some $h\in H$ so $\phi(h)=hg$ and we have shown surjectivity.
    \end{enumerate}
    Thus we conclude that phi is a bijection and that there exists a bijective mapping from a subgroup H to its right coset. The same can be proven for the left coset. Given that we call the number of distinct left cosets of $H$ in $G$ the index of $H$ in $G$, and there is a bijective mapping from H to the right coset and a biijective mapping from H to the left coset this implies that the right and left coset have the same cardinalities. In our context this relationship implies that the index definition extends to right cosets since we have demonstrated that the number of distinct left cosets is equal to the number of distinct right cosets.
  \end{proof}
  %%%%%%%%%%%%%%%%%%%%%%%%%%%%%%%%%%%%%%%%%%%%%%%%%%%%%%%%%%%%%%
 \block{Question #5}{Show that if $[G \, : \, H] = 2$ then $gH = Hg$ for all $g \in G$. }
 \begin{proof}
   $\newline$
Because the index of H is 2, there are just two left cosets  $$\{H, gH\}$$  and two right cosets $$\{H, H g\}$$Note that the disjoint union of $H$ and $gH$ is $G$.
\[
  G=H\cup gH
\]
However the disjoint union of $H$ and $Hg$ is also $G$.
\[
  G=H\cup Hg
\]
So $gH=G\cap H^c=Hg$.
 \end{proof}
\end{document}
