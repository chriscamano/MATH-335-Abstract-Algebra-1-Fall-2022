\documentclass[11pt]{article}
\usepackage{amssymb,latexsym,amsmath,amsthm,graphicx, cite}
\author{Chris Camano: ccamano@sfsu.edu}
\title{MATH 370  Homework 7 }
\date

\usepackage[many]{tcolorbox}
\tcbset{breakable}
\usepackage{mathptmx}
\usepackage{multirow}
\usepackage{float}
\restylefloat{table}
\hoffset=0in
\voffset=-.3in
\oddsidemargin=0in
\evensidemargin=0in
\topmargin=0in
\textwidth=6.5in
\textheight=8.8in
\marginparwidth 0pt
\marginparsep 10pt
\headsep 10pt

\theoremstyle{definition}  % Heading is bold, text is roman
\newtheorem{theorem}{Theorem}
\newtheorem{corollary}{Corollary}
\newtheorem{defn}{Definition}
\newtheorem{example}{Example}
\newtheorem{proposition}{Proposition}



\newcommand{\Z}{\mathbb{Z}}
\newcommand{\N}{\mathbb{N}}
\newcommand{\Q}{\mathbb{Q}}
\newcommand{\R}{\mathbb{R}}
\newcommand{\C}{\mathbb{C}}
\newcommand{\lcm}{\mathrm{lcm}}
\setlength{\parskip}{0cm}
%\renewcommand{\thesection}{\Alph{section}}
\renewcommand{\thesubsection}{\arabic{subsection}}
\renewcommand{\thesubsubsection}{\arabic{subsection}.\arabic{subsubsection}}
\bibliographystyle{amsplain}

%\input{../header}
\newcommand{\bigline}{\\\noindent\makebox[\linewidth]{\rule{\textwidth}{0.4pt}}\\}

\newcommand{\block}[2]{\begin{tcolorbox}[title={#1}]{#2}\end{tcolorbox}}
\begin{document}
\maketitle
\block{Definition}{
A group homomorphism is a function: \[
  \phi:(G,\circ)\rightarrow (H,*)
\]
$\phi$ is a homomorphism if $\forall a,b \in ( \phi(a*b))=\phi(a)\circ\phi(b)$
}

\block{Definition: Kernel of a homomorphism}{
\[
  ker(\phi)=\{g\in G : \phi(g)=e_H\}\subset G
\]
}

\textbf{Example: $\phi:S_3\mapsto\{1,-1\}$}
$ker(\phi)=\{\sigma\in S_3|\phi(\sigma)=1$

\block{Claim Given a homomorphism $\phi$ ker($\phi)$ is a normal subgroup of G\}}
{\begin{proof}
  First we prove that ker $\phi$ is a subroup of G, which means we must prove that if $a,b\in ker(\phi$) then $ab^{-1}\in ker(\phi)$ and $\phi(a*b^{-1})=e_H$
\end{proof}}
\begin{proof}
  $\phi(a*b^{-1})
=\phi(a)\circ \phi(b^{1-})$
\begin{align*}
  &\phi(a)\circ \phi(b^{1-})\\
  &e_H\circ e_H\\
  &e_H
\end{align*}
\end{proof}

\block{Proposition}{
ker$\phi$ is normal in (G,*)\\
we prove this if $\forall g\in G$ and $\forall a \in ker(\phi)$
\[
  gag^{-1}\in ker(\phi)
\]
\begin{align*}
  &\phi(gag^{-1})=\phi(g)\circ\phi(a)\circ\phi(g^{1-})\\
  &\phi(g)\circ e_H \circ\phi(g^{-1})\\
  &e_H
\end{align*}
So we have proven that $ker(\phi)$ is a normal subgroup in G.
}
Given a homomorphism the kernel of a homomorphism is normal in G. Given any normal subgroup there exists a homomorphism of phi sucht aht kernel of phi is equal to that normal subgroup:
\block{Theorem}{Every normal subgroup is a kernel of some homomorphism, this will be proven later}
\block{Theorem}{
 If $\phi(G,*)\rightarrow (H,\circ)$ is a homomorphism then $ker(\phi) =\{e_G\}$ if and only if $\phi$ is injective.
}
\begin{proof}
  We wish to show that the function is injective, thus for a,b, $\in G $  if \[
    \phi(a)=\phi(b)\mapsto a=b
  \]
  \\
\end{proof}


\end{document}
