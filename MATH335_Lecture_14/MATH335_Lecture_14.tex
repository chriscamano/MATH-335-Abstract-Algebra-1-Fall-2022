\documentclass[11pt]{article}
\usepackage{amssymb,latexsym,amsmath,amsthm,graphicx, cite}
 \author{Chris Camano: ccamano@sfsu.edu}
 \title{MATH 335  lecture 14 }
 \date

\usepackage{mathptmx}
\usepackage{multirow}
\usepackage{float}
\restylefloat{table}
\hoffset=0in
\voffset=-.3in
\oddsidemargin=0in
\evensidemargin=0in
\topmargin=0in
\textwidth=6.5in
\textheight=8.8in
\marginparwidth 0pt
\marginparsep 10pt
\headsep 10pt

\theoremstyle{definition}  % Heading is bold, text is roman
\newtheorem{theorem}{Theorem}
\newtheorem{corollary}{Corollary}
\newtheorem{defn}{Definition}
\newtheorem{example}{Example}
\newtheorem{proposition}{Proposition}



\newcommand{\Z}{\mathbb{Z}}
\newcommand{\N}{\mathbb{N}}
\newcommand{\Q}{\mathbb{Q}}
\newcommand{\R}{\mathbb{R}}
\newcommand{\C}{\mathbb{C}}

\newcommand{\lcm}{\mathrm{lcm}}



\setlength{\parskip}{0cm}
%\renewcommand{\thesection}{\Alph{section}}
\renewcommand{\thesubsection}{\arabic{subsection}}
\renewcommand{\thesubsubsection}{\arabic{subsection}.\ar  abic{subsubsection}}
\bibliographystyle{amsplain}

%\input{../header}

\newcommand{\bigline}{\\\noindent\makebox[\linewidth]{\rule{\paperwidth}{0.4pt}}\\}

\begin{document}
\maketitle
\section{Refresher on subgroups:}
\bigline

\defn Subgroup\\
Let G be a group. A subset H of G is called a subgroup if H itself is a group, when we restrict the group operation to H.\\
This is akin to saying:
\begin{enumerate}
  \item $e\in H$
  \item $\forall g_1,g_2 \in H,$ then $ g_1\circ g_2\in H $\\"Closed under the group operation of G"
  \item $\forall g\in H, g^{-1}\in H$\\
  Closed under taking inverses
\end{enumerate}
\bigline
\proposition A nonempty subset H of a group G is a subgroup if and only if $\forall g_1,g_2\in H, g_1g_2^{-1}\in H$ This satisfies the aformentined three critera needed to determine if something is a subgroup or not.
\begin{proof}
  Prove the identity element is in H.
Since H is not the empty set take any element in H. We also take: $g_1=g, g_2=g$ then :
\[
  g_1g_2^{-1}=gg^{-1}=e\in H
\]
Let $g\inH$ and take $g_1=e$, take $g_2=g$ then: $$g_1g_2^{-1}=eg^{-1}=g^{-1}\in H$$
Prove of property 2. : \\
Let $g,h\in H$
\[
  g_1=g,g_2=h^{-1}
\]
Thus \[
  g_1g_2^{-1}=g(h^{-1})^{-1}=gh\in H
\]
For the other direction of the biconditional we need to show that if all three properties are true then $g_1g_2^{-1}\in H$. However by the second proof we have that $g_2^{-1}\in H$ finally by the last proof we have $g_1g_2^{-1}\in H$
\end{proof}
\bigline
\defn Cyclic subgroups\\
Let G be a group, pick $g\in G$ now form the following set:
\[
  H=\{ g^k, k\in Z\}
\]
Here note that negative powers equate the powers over g inverse. H is a subgroup of G called the cyclic subgroup generated by g. Written conventionally as:
\[
  H=<g>
\]
the selected element g can be thought of as the seed or the generator of this set. \\ Comparable to a vector  subspace given a basis.
\begin{proof}
  We show that if $g_1,g_2\in <g>$ then $g_1g_2^{-1}\in<g>$\\
  let $g=g^i$ let $g_2=g^j$
  \[
    g_1g_2^{-1}=g^ig^{-j}=g^i-j\in <g>
  \]
\end{proof}
\bigline
Given G=$\mathbb{Z}$:
\[
  H<2>=\{2k:k\in \Z\}
\]
\[
  H<3>=\{3k:k\in \Z\}
\]
\[
  \mathbb{Z}=<1>
\]
\bigline
It is always true that
\[
  H<g>=H<g^{-1}>
\]
\bigline
If a group itself can be generated by an element of itself it is called a cyclic group.\\\\
Every subgroup of a cyclic group is cyclic.
\end{document}
