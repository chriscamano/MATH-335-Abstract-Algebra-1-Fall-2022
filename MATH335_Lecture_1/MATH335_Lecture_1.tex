\documentclass[12pt]{article}
\usepackage[pdftex]{graphicx}
\usepackage{amsmath}
\usepackage{amssymb}
\pagestyle{empty}
\author{Chris Camano: ccamano@sfsu.edu}
\title{MATH 335  Lecture 1 }
\date

\topmargin -0.6in
\headsep 0.40in
\oddsidemargin 0.0in
\textheight 9.0in
\textwidth 6.5in

\newcommand{\econst}{\mathrm{e}}
\newcommand{\diff}{\mathrm{d}}
\newcommand{\dwrt}[1]{\frac{\diff}{\diff #1}}
%%%%%%Macros for 425%%%%%%%%%%%%%%%%%%%
\newcommand{\q}{\quad}
\newcommand{\tab}{\\\\}
\renewcommand{\labelenumi}{\alph{enumi})}
\newcommand{\sect}[1]{\section*{#1}}

%%%%%%Vector Spaces%%%%%%%%%%%%%%%%%%%
\newcommand{\R}{\mathbb{R}}
\newcommand{\C}{\mathbb{C}}
\newcommand{\F}{\mathbb{F}}
\newcommand{\rtwo}{\mathbb{R}^2}
\newcommand{\mxn}{{mxn}}

%%%%%%Sets and common phrases%%%%%%%%%
\newcommand{\Axb}{\textbf{Ax=b} }
\newcommand{\Axz}{\textbf{Ax=0} }
\newcommand{\dim}{\text{dim}}
\newcommand{\lc}{linear combination }
\newcommand{\let}{\text{Let }}
\newcommand{\tf}{\therefore}
%%%%%%%%%Analysis%%%%%%%%%%%%%%%%%%%%%
\newcommand{\arr}{\rightarrow}
\newcommand{\xlim}{\lim_{x\rightarrow \infty}}
\newcommand{\Z}{\mathbb{Z}}
\newcommand{\N}{\mathbb{N}}
\newcommand{\ep}{\epsilon}
\newcommand{\i}{\text{ if }}
\newcommand{\and}{\text{ and }}


\everymath={\displaystyle}


\begin{document}
\maketitle
\sect{Number systems}
\textbf{The Integers }\\
\[
  \Z=\{...,-1,0,1,...\}
\]
\textbf{The Natural Numbers}\\
\[
  \mathbb{N}=\{1,2,3,...\}
\]
\\
\\
\textbf{Definition}: Well ordered\\
A non empty subset S of integers is wellordered if and only if S contains a smallest or least element. \\\\
\textbf{Well ordering of $\N$} Every non empty subset of $\N$ is well ordered. This is to say it has a smallest element since it will always contain positive integer values. \\\
\sect{Divisibility}\\
Let $A \neq 0, b\in \Z$ a divides b if and only if b can be expressed in the following way:
\[
  b=ak,k \in \Z
\]
a is a divisor of b and is often denoted in the following way, read as a divides b:
\[
   a|b
\]\\
\textbf{Properties of divisibility}\\
Division is transitive, therefore:
\[
  \text{if } a|b \land a|c \arr a| b \pm c
\]
if $a|n$ and $a|(n+m)$ then $a|m$. \\
\sect{Prime Numbers}\\
\textbf{Definition}:\\
A positive integer p $>1$ is said to be a prime number if and only if the one only positive integer divisiors are 1 and p. \\\\
\textbf{Fundemental Theorem of Arithmetic}\\
Every positive integer $n>1$ is equal to:
\[
  n=P_1^{\alpha_1}P_2^{\alpha_2}\cdots P_{k-1}^{\alpha_{k-1}}P_k^{\alpha_k}=\prod_{i=1}^kP_i^{\alpha_i}
\]
Where $P_1,..,P_k$ are distinct primes and $\alpha_1,..,\alpha_k$ are positive integers. These two sets of primes and positive integers are unique to the prime factorization of a given number.\\\\
\textbf{Proposition}:\\
Let p be a prime number and a and b$\in \Z$ if $p|ab$ then $p|a$ or $p|b$\\\\
\textbf{Greatest Common Denominator}:\\
Let d be a positive integer, d is the greatest common divisor(gcd) of a,b$\in \Z$ if and only if two criteria are satisfied:
\[
  (1)\quad d|a \text{ and } d|b
\]
\[
  (2)\text{ if } e|a \text{ and } e|b \text{ then } e|d
\]
Statement one states that the given positive integer must in fact divide both a and b. \\
Statement two says that if there exists another comon divisor then it must also divide d. This is a statement asserting that no greater common divisor exists.
\end{document}
