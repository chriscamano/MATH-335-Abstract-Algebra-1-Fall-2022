\documentclass[11pt]{article}
\usepackage{amssymb,latexsym,amsmath,amsthm,graphicx, cite}
\usepackage{hyperref}
\hypersetup{
     colorlinks=true,
     linkcolor=blue,
     filecolor=blue,
     citecolor = black,      
     urlcolor=blue,
     }
%\usepackage[sc]{mathpazo}
%\linespread{1.05}         % Palatino needs more leading (space between lines)
%\usepackage[T1]{fontenc}
\usepackage{mathptmx}
\usepackage{multirow}
\usepackage{float}
\restylefloat{table}
\hoffset=0in 
\voffset=-.3in
\oddsidemargin=0in
\evensidemargin=0in
\topmargin=0in 
\textwidth=6.5in
\textheight=8.8in
\marginparwidth 0pt
\marginparsep 10pt
\headsep 10pt

\theoremstyle{definition}  % Heading is bold, text is roman
\newtheorem{theorem}{Theorem}
\newtheorem{corollary}{Corollary}
\newtheorem{definition}{Definition}
\newtheorem{example}{Example}
\newtheorem{proposition}{Proposition}



\newcommand{\Z}{\mathbb{Z}}
\newcommand{\N}{\mathbb{N}}
\newcommand{\Q}{\mathbb{Q}}
\newcommand{\R}{\mathbb{R}}
\newcommand{\C}{\mathbb{C}}

\newcommand{\lcm}{\mathrm{lcm}}



\setlength{\parskip}{0cm}
%\renewcommand{\thesection}{\Alph{section}}
\renewcommand{\thesubsection}{\arabic{subsection}}
\renewcommand{\thesubsubsection}{\arabic{subsection}.\arabic{subsubsection}}
\bibliographystyle{amsplain} 

%\input{../header}


\begin{document}

%\homework{}{Homework V}

\begin{enumerate}

\item Consider the following permutations in $S_{15}$:
  $$ \sigma = \left( \begin{array}{ccccccccccccccc} 1 & 2 & 3 & 4 & 5 & 6 & 7 & 8 & 9 & 10 & 11 & 12 & 13 & 14 & 15 \\
                       13& 2 &15&14&10& 6 &12& 3 & 4 &   1 &   7 &   9 &   5 & 11 &   8 \end{array} \right) $$
  $$      \tau = \left( \begin{array}{ccccccccccccccc} 1 & 2 & 3 & 4 & 5 & 6 & 7 & 8 & 9 & 10 & 11 & 12 & 13 & 14 & 15 \\
                        14& 9 &10& 2 &12 & 6 & 5 & 11 & 15 &   3 &   8 &   7 &   4 & 1 &   13 \end{array} \right). $$
Compute $\sigma^2$, $\sigma \tau$, $\tau \sigma$, $\tau^2\sigma$, and $\sigma^{-1} \tau$. 
\item Here I will introduce a concept that we will take up very soon. Let $G$ be  group and let $g \in G$ be an element of the group. The {\it order} of $g$, denoted by $|g|$,
  is the smallest positive number $n$ such that $g^n = e$. If there is no such $n$, we say that $g$ has infinite order.
  \begin{itemize}
     \item[a)] In any group, what is $|e|$ ?
     \item[b)] Compute the order of the elements in $\Z_6$.
     \item[c)] Compute the order of the elements in $U(9)$.
     \item[d)] Compute the order of the elements in $S_3$.
     \item[e)] Find an element of $GL_2$ that has infinite order. Justify your choice.
      \item[f)] Find an element of order $6$ in $S_5$. 
  \end{itemize}
\item We showed that $S_3$ is a non-abelian group since we identified two particular elements $\sigma$ and $\tau$ in $S_3$ such that $\sigma \tau \neq \tau \sigma$.
 Prove that $S_n$ is non-abelian for all $n\geq 3$. [use $\sigma$ and $\tau$ !]
\item Exercise 2 in Section 3.5 of our textbook. 
\item Let $H = \{ 2^k \, :\, k \in \Z\}$. Show that $H$ is a subgroup of $\Q^*$ (i.e. the group of nonzero rational numbers under multiplication).
\item Let $n$ be any positive integer and let $n\Z = \{kn \, : \, k \in \Z\}$ be the set of all integer multiples of $n$. Show that $n\Z$ is a subgroup of $\Z$.
\item Prove that the following set of matrices
  $$ \left\{ \left( \begin{array}{ccc} 1 & x & y \\ 0 & 1 & z  \\ 0 & 0 & 1 \end{array} \right) \, : \,  x, y, z \in \R \right\} $$
  is a subgroup of $GL_3$. 


 \end{enumerate}



\end{document}



