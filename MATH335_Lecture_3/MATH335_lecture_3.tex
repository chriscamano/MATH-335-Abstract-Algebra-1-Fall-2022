\documentclass[12pt]{article}
\usepackage[pdftex]{graphicx}
\usepackage{amsmath,amssymb,amsthm}
\usepackage{hyperref}
\pagestyle{empty}
\author{Chris Camano: ccamano@sfsu.edu}
\title{MATH 370  Lecture 3 }
\date

\topmargin -0.6in
\headsep 0.40in
\oddsidemargin 0.0in
\textheight 9.0in
\textwidth 6.5in
\vfuzz2pt
\hfuzz2pt

%%%%Short cuts and formatting%%%%%%%%%%
\newcommand{\q}{\quad}
\newcommand{\tab}{\\\\}
\renewcommand{\labelenumi}{\alph{enumi})}
\newcommand{\sect}[1]{\section*{#1}}

%%%%%%Vector Spaces%%%%%%%%%%%%%%%%%%%
\newcommand{\R}{\mathbb{R}}
\newcommand{\C}{\mathbb{C}}
\newcommand{\F}{\mathbb{F}}
\newcommand{\Q}{\mathbb{Q}}
\newcommand{\rtwo}{\mathbb{R}^2}
\newcommand{\mxn}{{m\times n}}

%%%%%%Sets and common phrases%%%%%%%%%
\newcommand{\Axb}{\textbf{Ax=b} }
\newcommand{\Axz}{\textbf{Ax=0} }
\newcommand{\dim}{\text{dim}}
\newcommand{\lc}{linear combination }
\newcommand{\let}{\text{Let }}
\newcommand{\tf}{\therefore}
%%%%%%%%%Analysis%%%%%%%%%%%%%%%%%%%%%
\newcommand{\arr}{\rightarrow}
\newcommand{\xlim}{\lim_{x\rightarrow \infty}}
\newcommand{\Z}{\mathbb{Z}}
\newcommand{\N}{\mathbb{N}}
\newcommand{\ep}{\varepsilon}
\newcommand{\i}{\text{ if }}
\newcommand{\and}{\text{ and }}
%%%%%% Theorem formatting%%%%%%%%%%%
\newtheorem{thm}{Theorem}[section]
\newtheorem{cor}[thm]{Corollary}
\newtheorem{lem}[thm]{Lemma}
\newtheorem{prop}[thm]{Proposition}
\theoremstyle{definition}
\newtheorem{defn}[thm]{Definition}
\theoremstyle{remark}
\newtheorem{rem}[thm]{Remark}
\numberwithin{equation}{section}
\everymath={\displaystyle}


\begin{document}
\maketitle

\sect{Opening notes}\\
The following is a recap of a short lecture on how to compute the greatest common denominator and least common multiple of two very large integers. \\\\
\thm Division Algorithm:
Let $a,b\in \Z, b>0$ then:
\[
  \exists !\quad q,r \in \Z : a=qb+r\quad 0\leq r < b
\]
\begin{proof}
  Let S be the following nonempty set:
  \[
    S=\{a-bk: k \in \Z \land a-bk \geq0\}
  \]
  First show that s is nonempty: If $a\geq 0$ then for k=0$ a-bk=a$ which means a is in the set. \\\\
  If $a<0$ then let k=2a therefore $a-b(2a)=a(1-2b)$ so for all b since a is negative we get a product of two negative numbers which is postiive. \\\\
  If $0 \in S $ then $\exists q \in \Z$ such that $a-bq=0$. This means that $a=bq+0$ or $a=bq$ \\\\
If $0 \notin S$ since S is a nonempty set of positive integers by the well ordering principle there exists a smallest element of that set since the set S is then a subset of the natural number . let this smallest element be r. $r\in S, r\in \Z^+, r=\min(S)$\\\\
So there exists an integer q such that $r=a-bq$ this implies that $a=bq+r$ \\\\
We now need to show that r is smaller than b: Suppose that r is not smaller then b, this would imply that $r\geqb$ which means that $a=bq+r$ but since r is greater than b we can right the expression as $b(q+1)+(r-b)$ but since r is greater than b $ r-b$ is non negative. $r-b \geq 0$ Hence r-b should be in s however r-b is smaller r but r is supposed to be the smallest element in S by the well ordering principle.
\end{proof}
\end{document}
