\documentclass[11pt]{article}
\usepackage{amssymb,latexsym,amsmath,amsthm,graphicx, cite}
 \author{Chris Camano: ccamano@sfsu.edu}
 \title{MATH 335  Lecture 6 }
 \date

\usepackage{mathptmx}
\usepackage{multirow}
\usepackage{float}
\restylefloat{table}
\hoffset=0in
\voffset=-.3in
\oddsidemargin=0in
\evensidemargin=0in
\topmargin=0in
\textwidth=6.5in
\textheight=8.8in
\marginparwidth 0pt
\marginparsep 10pt
\headsep 10pt

\theoremstyle{definition}  % Heading is bold, text is roman
\newtheorem{theorem}{Theorem}
\newtheorem{corollary}{Corollary}
\newtheorem{definition}{Definition}
\newtheorem{example}{Example}
\newtheorem{proposition}{Proposition}



\newcommand{\Z}{\mathbb{Z}}
\newcommand{\N}{\mathbb{N}}
\newcommand{\Q}{\mathbb{Q}}
\newcommand{\R}{\mathbb{R}}
\newcommand{\C}{\mathbb{C}}

\newcommand{\lcm}{\mathrm{lcm}}



\setlength{\parskip}{0cm}
%\renewcommand{\thesection}{\Alph{section}}
\renewcommand{\thesubsection}{\arabic{subsection}}
\renewcommand{\thesubsubsection}{\arabic{subsection}.\arabic{subsubsection}}
\bibliographystyle{amsplain}

%\input{../header}


\begin{document}

\maketitle
\prop Let m and n be two positive integers that are relativley prime, this is to say that gcd(m,n)=1\\ Then :
\[
  \phi(mn)=\phi(m)\phi(n)
\]
\begin{proof}
  let:
  \[
    A=\{1\leq a\leq m: gcd(a,n)=1\}
  \]
  This is equivilant to saying the set of all integers from 1 to m that are relativley prime to m. The cardinality of A is equal to $\phi(m)$
\\
\[
  B=\{1 \leq b\leq n: gcd(b,n)=1\}
\]
This is equivilant to saying the set of all integers from 1 to n that are relativley prime to n. The cardinality of B is equal to $\phi(n)$ \\
\[
  C=\{1\leq c \leq mn: gcd(c, mn )=1 \}
\]
This is equivilant to saying the set of all integers from 1 to mn that are relativley prime to mn. The cardinality of C is equal to $\phi(mn)$ \\
Observe the following:
\[
  |Ax B|=|\{(a,b):a\in A, b\in B\}|=|A||B|=\phi(m)\phi(n)
\]
if we can then show that the cardinality of C is equal to the cardinality of the cartesian product then we have proven equivilancy. To prove that two sets have the same cardinality we typically prove that there is a bijection $\Psi$ between the two sets. \\\\
If we can show $\exists \quad \Psi : C\mapsto AxB$ then we prove the original problem. \\\\
Given integers c and m we know that we can divide one by the other and obtain a remainder by the divison algorithm. such that \[
  c=qm +r\quad 0\leq r < m
\]
denote r as $\bar{c_m}$\\
for $c\in C$
\[
  \Psi(c)=(\bar{c_m},\bar{c_n})
\]
$0\leq \bar{c_m}< m $ by definition of remainder , however $\bar{c_m}$ cannot be equal to 0 by the definition of set c so : $1\leq \bar{c_m}< m $
also  $1\leq \bar{c_n}< n$. We now need to show that $gcd(\bar{c_m}),m=1, gcd(\bar{c_n},n=1))$ to prove they belong to A and B. \\\\
if $gcd(\bar{c_m},=d>1$ then $d|\bar{c_m}$ and $d|m$ therefore $d|c$ . Since d divides m and d divides c then this would imply that gcd(c,m)>1, but this implies gcd(c,mn)=1 which contradicts the set that c was chosen from, that being C which states that gcd must be equal to 1; We can extend this to n and $\bar{c_n}$ with a symmetric proof. \\\\
\textbf{Injectivity}\\
We now demonstrate that $\Psi(c)$ is injective suppose$c_q, c_2\in C$ and $\Psi(c_1)=\Psi(c_2)$ we then need to conclude that this implies that $c_1$ and $c2$ are the same. In other words we need to prove that for each element in the domain there exists a uniuq element in the codomain. \\\\
\[
  \Psi(c_1)=(\bar{c_{1,m}},\bar{c_{1,n}})\quad \Psi(c_2)=(\bar{c_{2,m}},\bar{c_{2,n}})
\]
\[
  c_1=mq_1+\bar{c_{1,m}}\quad c_2=mq_2+\bar{c_{2,m}}
\]
we need to now show that $c_1$ and $c_2$ are the same. Let us start by subtracting these two expressions: When assuming that the remainders are the same this gives
\begin{align*}
  &c_1-c_2=m(q_1-q_2)\\
  &m|c_1-c_2
\end{align*}
likewise:
\begin{align*}
  &c_1-c_2=n(q^*_1-q^*_2)\\
  &n|c_1-c_2
\end{align*}
By definition m and n are relativley prime. Since gcd(m,n)=1 then $mn|c_1-c_2$ By construction we have: \[
  1 \leq c_1<mn\quad 1 \leq c_2 <mn
\]
The only time that this is true is if the difference is equal to zero since otherwise the difference of the two chosen c will be less than mn. therefore we have proven injectivity and that $c_1=c_2$\\\\
\textbf{Surjectivity }\\
To show that $\Psi$ is surjective we start with an element in the codomain and create a corresponding element in the domain. Take:
\[
  (a,b)\in A\cross B\quad 1\leq a \leq m, gcd(a,m)=1\quad 1\leq b\leq m gcd(b,m)=1
\]
We must now construct $c\in C, q\leq c \leq mn , gcd(c,mn=1)$ \\
Such that \[
  \Psi(c)=(a,b)
\]
Since gcd(m,n)=1 then we can express 1 as a linear combination of m,n as such:
\[
  \exists \quad t,u\in \Z: mu+nt=1\\
\]
What would the word munt mean ? mean runt?\\
Let z= an(t)+bmu. This is some linear combination using the coefficients of the gcd of c and the integers a,b. \\\\
Conjecture: z mod m =a . \\
\[
  z=a(1-mu)+bmu=(bu-au)m+a
\]
so we have $\bar{z_m}=a$ and $\bar{z_n}=b$, however we now need to determine the size of z. \\\\
Now we let c to be equal to the remainder we get when we divide z by mn. so:
\[
  0\leq c <mn
\]
\[
  z=qmn+c
\]
since z mod m is equal to a then if divide c by m we get  remainder a since qmn must be equal to zero which implies that c is the term needed to satisfy $\bar{z_m}=a$ likewise for b
Finally we need to show tha c is in c the only criterion missing is that gcd(c,mn)=1.
If we can prove that gcd(z,mn)=1 then we can show that gcd(c,mn)=1 by the property that gcd(a,b)=gcd(b,r) when a=bq+r,\\\\
To show that consider the following:
\[
  gcd(a,m)=gcd(c,m)=1\quad  gcd(b,n)=gcd(c,n)=1 \therefore gcd(c,mn=1)
\]
And we have proven surjectivity



\end{proof}

\defn Equivilance realtions and equivilance classes\\
An equivilance relation on a set X is a subset R as follows:
\[
  R\subset X \times X
\]
such that
\begin{enumerate}
  \item $(x,x)\in R \quad \forall x\in X$ (Reflexivity)
  \item $(x,y)\in R \iff (y,x)\in R$ (Symmetric)
  \item if $(x,y)\in R$ and $(y,z)\in R$ then $(x,z)\in R$ (Transitivity)
\end{enumerate}
Instead of denoting $(x,y)\in R$ we write $x\sim y$. with this new notation:
\begin{enumerate}
  \item x\sim x
  \item x\sim y $\iff$ y\sim x
  \item x\sim y and y\sim z then x\sim z
\end{enumerate}

\end{document}
