\documentclass[11pt]{article}
\usepackage{amssymb,latexsym,amsmath,amsthm,graphicx, cite}
 \author{Chris Camano: ccamano@sfsu.edu}
 \title{MATH 335  lecture 5 }
 \date

\usepackage{mathptmx}
\usepackage{multirow}
\usepackage{float}
\restylefloat{table}
\hoffset=0in
\voffset=-.3in
\oddsidemargin=0in
\evensidemargin=0in
\topmargin=0in
\textwidth=6.5in
\textheight=8.8in
\marginparwidth 0pt
\marginparsep 10pt
\headsep 10pt

\theoremstyle{definition}  % Heading is bold, text is roman
\newtheorem{theorem}{Theorem}
\newtheorem{corollary}{Corollary}
\newtheorem{definition}{Definition}
\newtheorem{example}{Example}
\newtheorem{proposition}{Proposition}



\newcommand{\Z}{\mathbb{Z}}
\newcommand{\N}{\mathbb{N}}
\newcommand{\Q}{\mathbb{Q}}
\newcommand{\R}{\mathbb{R}}
\newcommand{\C}{\mathbb{C}}

\newcommand{\lcm}{\mathrm{lcm}}



\setlength{\parskip}{0cm}
%\renewcommand{\thesection}{\Alph{section}}
\renewcommand{\thesubsection}{\arabic{subsection}}
\renewcommand{\thesubsubsection}{\arabic{subsection}.\arabic{subsubsection}}
\bibliographystyle{amsplain}

%\input{../header}


\begin{document}

\maketitle
\defn Euler's $\phi$ function:
The euler's phi function can be defined in the following way. \\\\
Take any positive integer. Then $\phi (n)=|\{a\in \Z: q\leq a \leq n, gcd(a,n)=1\}$
Phi of n counts the number of integers between one and n that are relatively prime to n. \\\\
if p is prime then#$ \phi(p)=p-1$#\\\\
if p is a prime number and k is a postive integer then:
\[
  \phi(p^k)=p^k-p^{k-1}=p^{k-1}(p-1)
\]
\begin{proof}
  \begin{align*}
    &\phi(p^k)=|\{a\in \Z q\leq a\leq p^k,gcd(a,p^k)=1\}|\\
    &p^k-|b\in \Z: 1\leq b\leq p^k, gcd(b,p^k)\neq 1\}|\\
    &p^k-|b\in \Z: 1\leq b\leq p^k, gcd(b,p^k)= kp,k\in \Z\}|\\
    &p^k-|\{p,2p,3p,...,p^{k-1}p\}| \\
    &p-p^{k-1}\\
    &p^{k-1}(p-1)
  \end{align*}
\end{proof}
If the gcd of a number and a prime power is not equal to one then the gcd must be a multiple of 1. \\\\
\prop If $m,n\in \mathbb{Z}^+, gcd(m,n)=1$ then :
\[
  \phi(mn)=\phi(m)\phi(n)
\]\\\\
\[
  \phi(n)=\prod_{i=1}^k\phi(p_i^{\alpha_i})=\prod_{i=0}^kp_i^{\alpha_i-1}(p_i-1)
\]
\[
  \phi(nm)=\prod_{i=1}^k\phi(p_i^{\alpha_i})\prod_{i=1}^k\phi(p_i^{\beta_i})=\prod_{i=1}^kq_i^{\beta_i-1}(q_i-1)p_i^{\alpha_i-1}(p_i-1)=
\]
\begin{proof}
  If t is a postive integer and $\prod^kp_i$ are the prime divisiors of t then :
  \[
    \phi(t)=\prod_{i=1}^k\phi(p_i^{\alpha_i})=t-|\{1\leq a \leq t: gcd(a,t)\neq 1\}|
  \]
  \[
      \phi(t)=\prod_{i=1}^k\phi(p_i^{\alpha_i})=t-|\{1\leq a \leq t: \exists p_i: p_i|a}|
  \]
  where $p_i$ is one of t's prime factors:
  Since gcd =1 then:
  \[
    \{\prod_{i=1}^k\phi(p_i^{\alpha_i})\}\cap \{\prod_{i=1}^k\phi(q_i^{\beta})\}=\emptyset
  \]
  \[
    \phi(mn)=mn-|\{1\leq a \leq mn: p_i|a \lor q_j|a\}|
  \]
  This is equivilant to :
  \[
    mn-|\{1\leq b\leq m : \text{ b is divisble by at least one prime factorof n }|n-|\{1\leq c\leq n : \text{ c is dibisble by at least one prime factor of m} \}|m+|\{1\leq b\leq m : \text{ b is divisble by at least one prime factorof n }\}||\{1\leq c\leq n : \text{ c is dibisble by at least one prime factor of m} \}|
  \]
\end{proof}
\end{document}
