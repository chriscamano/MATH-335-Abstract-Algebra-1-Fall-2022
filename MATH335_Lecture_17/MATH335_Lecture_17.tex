\documentclass[11pt]{article}
\usepackage{amssymb,latexsym,amsmath,amsthm,graphicx, cite}
\author{Chris Camano: ccamano@sfsu.edu}
\title{MATH 335  lecture 17 }
\date

\usepackage[many]{tcolorbox}
\tcbset{breakable}
\usepackage{mathptmx}
\usepackage{multirow}
\usepackage{float}
\restylefloat{table}
\hoffset=0in
\voffset=-.3in
\oddsidemargin=0in
\evensidemargin=0in
\topmargin=0in
\textwidth=6.5in
\textheight=8.8in
\marginparwidth 0pt
\marginparsep 10pt
\headsep 10pt

\theoremstyle{definition}  % Heading is bold, text is roman
\newtheorem{theorem}{Theorem}
\newtheorem{corollary}{Corollary}
\newtheorem{defn}{Definition}
\newtheorem{example}{Example}
\newtheorem{proposition}{Proposition}



\newcommand{\Z}{\mathbb{Z}}
\newcommand{\N}{\mathbb{N}}
\newcommand{\Q}{\mathbb{Q}}
\newcommand{\R}{\mathbb{R}}
\newcommand{\C}{\mathbb{C}}
\newcommand{\lcm}{\mathrm{lcm}}
\setlength{\parskip}{0cm}
%\renewcommand{\thesection}{\Alph{section}}
\renewcommand{\thesubsection}{\arabic{subsection}}
\renewcommand{\thesubsubsection}{\arabic{subsection}.\arabic{subsubsection}}
\bibliographystyle{amsplain}

%\input{../header}
\newcommand{\bigline}{\\\noindent\makebox[\linewidth]{\rule{\textwidth}{0.4pt}}\\}

\newcommand{\block}[2]{\begin{tcolorbox}[title={#1}]{#2}\end{tcolorbox}}
\begin{document}
\maketitle

\block{Recap of main ideas relating to cyclic groups}{
\begin{enumerate}
  \item If G=$<a>$ is a cyclic group then every subgroup H of G is also a cyclic group\\
  In fact H=$<a^d>$ where d is the smallest positive integer such that:
  \[
    a^d\in H
  \]
  \item If an element has finite order, this is to say: $|a|=n$ then:
  \[
    |a^k|=\frac{|a|}{\gcd(n,k)}
  \]
  This is to say: If G is a cylic group generated by a where : $|a|=n$ then $|G|=n$. Then the order of H=$<a^k>$ is $\frac{|a|}{\gcd(n,k)}$
\end{enumerate}
}
\proposition Let G=$<a>$ be a cyclic group of order n. Then for every positive divisor d of n there exists a subgroup of order $\frac{n}{d}$ \\
\begin{proof }
  Let $H=<a^d>$ then
  \[
    |H|=|a^d|=\frac{n}{\gcd{d,n}}=\frac{n}{d}
  \]
\end{proof }
\block{Theorem}{
\Theorem Let G be a cyclic group of order n\\
Then for every positive divisor d of n there is a unique subgroup of order $\frac{n}{d}$\\}
\begin{proof}
Let $d|n$ then $H<a^d>$ is a subgroup of order $\frac{n}{d}$  We now  need to show that H is the only unique subgroup of $\frac{n}{d}$\\
Suppose there exists another subgroup $K=<a^b>$ with equivilant order: $\frac{n}{d}$ \\
\[
  |K|=\frac{n}{\gcd(b,n)}=\frac{n}{d}
\]
Thus we conclude that the $gcd(b,n)=d$\\
This implies that $d|b$ so:\\
\textbf{Claim}:
$K=<a^b>\subset H=<a^d>$

\[
  a^b=(a^d)^k, k \in \Z\in H
\]
This implies that K is a subgroup of G and that $K<=H<=G$ since $|H|=|K|=\frac{n}{d}\rightarrow H=K$
\end{proof}
Prime decomposition can be read off of a divisibility lattice by taking the proudct over the subgroups connected to the identity element and taking exponents associated with the number of connections within the lattice at each prime ordered subgroup.
\block{new idea?}{The divisibility lattice can be formed by reverse engineering the prime decomposition\\}\\
\block{Definition}{
Left cosets of a subgroup\\
Let G be a group. and let H be a subgroup of G. Then the left Coset with representative $g\in G$ is a set $gH=\{gh:h\in H \}$
\\\\
Right coset of a subgroup\\
Let G be a group. and let H be a subgroup of G. Then the right Coset with representative $g\in G$ is a set $Hg=\{hg:h\in H \}$
}
The representatives of a coset are always the elemenents of a coset regardless of whether or not the group is cyclic.
\end{document}
