\documentclass[11pt]{article}
\usepackage{amssymb,latexsym,amsmath,amsthm,graphicx, cite}
\author{Chris Camano: ccamano@sfsu.edu}
\title{MATH 335  Lecture 22 }
\date

\usepackage[many]{tcolorbox}
\tcbset{breakable}
\usepackage{mathptmx}
\usepackage{multirow}
\usepackage{float}
\restylefloat{table}
\hoffset=0in
\voffset=-.3in
\oddsidemargin=0in
\evensidemargin=0in
\topmargin=0in
\textwidth=6.5in
\textheight=8.8in
\marginparwidth 0pt
\marginparsep 10pt
\headsep 10pt

\theoremstyle{definition}  % Heading is bold, text is roman
\newtheorem{theorem}{Theorem}
\newtheorem{corollary}{Corollary}
\newtheorem{defn}{Definition}
\newtheorem{example}{Example}
\newtheorem{proposition}{Proposition}



\newcommand{\Z}{\mathbb{Z}}
\newcommand{\N}{\mathbb{N}}
\newcommand{\Q}{\mathbb{Q}}
\newcommand{\R}{\mathbb{R}}
\newcommand{\C}{\mathbb{C}}
\newcommand{\lcm}{\mathrm{lcm}}
\setlength{\parskip}{0cm}
%\renewcommand{\thesection}{\Alph{section}}
\renewcommand{\thesubsection}{\arabic{subsection}}
\renewcommand{\thesubsubsection}{\arabic{subsection}.\arabic{subsubsection}}
\bibliographystyle{amsplain}

%\input{../header}
\newcommand{\bigline}{\\\noindent\makebox[\linewidth]{\rule{\textwidth}{0.4pt}}\\}

\newcommand{\block}[2]{\begin{tcolorbox}[title={#1}]{#2}\end{tcolorbox}}
\begin{document}
\maketitle
\block{Normal Subgroups/Groups}{
A normal group or subgroup is a group where the left cosets equals the right cosets for all reprsentatives in a general group. \\\\
Let G be a group a subgroup of N of G is called normal if $gN=Ng, \forall g\in G$
}
Exapmple: $<(1,23)>=N$ generates a 3 element subgroup with 2 distinct cosets. N is a normal subgroup since eN=Ne, $\sigma N=N\sigma$
\\
Note that the subgroup need not be abelian to be normal
\block{Proposition: If G is an abelian group thene very subgroup of G is a normal subgroup}{\begin{proof}
  Let H be a subgroup of G, for $g\in G$ that $gH=Hg$ however since g is abelian this fact is quite obvious.
  \[
    gH=\{gh:h\in H\}=\{hg:h\in H\}=Hg
  \]

\end{proof}


}
\block{We will see that $A_n$ is a normal subgroup of $S_n$ but it is neither abelian nor cyclic}{}
\block{Definition: Let H be a subgroup of group G, and let$g\in G$}{ Consider the set $$gHg^{-1}=\{ghg^{-1}:h\in H$$}
Example\\
Let G=$S_3#: H=<\sigma>=\{e,(1\2)$
\[
  g=(1\ 2\ 3)\quad gHg^{-1}=(1\ 2\ 3)H(1\ 3\ 2)^{-1}=\{(1\ 2\ 3)e, (1\ 2\ 3)(12)(1\ 3\ 2)\}=\{e,(2 \ 3)\}
\]
\block{Let N be a sugroup of a group G then the following are equivilant }{\begin{enumerate}
  \item N is Normal
  \item$ gNg^{-1}\subset N \quad \forall \ g\in G$
  \item  $gNg^{-1}=N \quad \forall\ g \in G $
\end{enumerate}
\begin{proof}
  we prove that (1) implies(2) implies(3):\\
  \textbf{(1) implies (2)}:\\
  Since N is a normal subgroup this implies that $gN=Ng$ for all elements g in G. \\
  \begin{align*}
    &gn\in gN, \rightarrow gn\in Ng\\
    &\therefore gn=n^*g \text{ for }n^*\in N\\
    &gng^{-1}=n^*\rightarrow gng^{-1}\in N
  \end{align*}
  \textbf{(2) implies (3)}:\\
  We wish to show that $  gNg^{-1}\subset N\rightarrow gNg^{-1}=N\quad \forall\ g\in G $\\
  We need to show the other side of the set inclusion thus we prove:
  \[
    N\subset gNg^{-1}
  \]
  let n$\in N$, consider $g^{-1}n(g^{-1})^{-1}=g^{-1}ng\in N $\\ by the previous proof, therefore we sho that $g^{-1}ng=n^*$ for $n^*\in N$ \\
  \[
    g^{-1}ng=n^*\rightarrow n=gn^*g^{-1}
  \]
  So we have shown that N is a contained in $gNg^{-1}$ and we conclude that they are equal

  \textbf{(3) implies (1)}:\\
      $gNg^{-1}=N\quad \forall\ g\in G \rightarrow gN=Ng$\\
      We show this by a set inclusion proof as follows:
      \\\textbf{gN$\subset$Ng}\\
      $gn\in gN$\\
      $gng^{-1}\in gNg^{-1}=N\rightarrow gng^{-1}=n^*, n^*\in N$\\ Right multiplication yeilds:
      \[
        gn=n^*g
      \]\\
      And we conclude $gN\subset Ng$
      \textbf{Ng$\subset$gN}\\
      This proof is symmetric
\end{proof}}
\block{Theorem: G is always a normal subgroup of itself. }{}
\block{Start with a group and a subgroup H of G }{Consider the set of all distinct left cosets of H in G.}
\end{document}
