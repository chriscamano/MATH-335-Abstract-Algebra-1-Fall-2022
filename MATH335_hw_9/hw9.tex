\documentclass[11pt]{article}
\usepackage{amssymb,latexsym,amsmath,amsthm,graphicx, cite}
\usepackage{hyperref}
\hypersetup{
     colorlinks=true,
     linkcolor=blue,
     filecolor=blue,
     citecolor = black,      
     urlcolor=blue,
     }
%\usepackage[sc]{mathpazo}
%\linespread{1.05}         % Palatino needs more leading (space between lines)
%\usepackage[T1]{fontenc}
\usepackage{mathptmx}
\usepackage{multirow}
\usepackage{float}
\restylefloat{table}
\hoffset=0in 
\voffset=-.3in
\oddsidemargin=0in
\evensidemargin=0in
\topmargin=0in 
\textwidth=6.5in
\textheight=8.8in
\marginparwidth 0pt
\marginparsep 10pt
\headsep 10pt

\theoremstyle{definition}  % Heading is bold, text is roman
\newtheorem{theorem}{Theorem}
\newtheorem{corollary}{Corollary}
\newtheorem{definition}{Definition}
\newtheorem{example}{Example}
\newtheorem{proposition}{Proposition}



\newcommand{\Z}{\mathbb{Z}}
\newcommand{\N}{\mathbb{N}}
\newcommand{\Q}{\mathbb{Q}}
\newcommand{\R}{\mathbb{R}}
\newcommand{\C}{\mathbb{C}}

\newcommand{\lcm}{\mathrm{lcm}}



\setlength{\parskip}{0cm}
%\renewcommand{\thesection}{\Alph{section}}
\renewcommand{\thesubsection}{\arabic{subsection}}
\renewcommand{\thesubsubsection}{\arabic{subsection}.\arabic{subsubsection}}
\bibliographystyle{amsplain} 

%\input{../header}


\begin{document}

%\homework{}{Homework IX}

\begin{enumerate}

\item Consider the following permutations in $S_{15}$:
  $$ \sigma = \left( \begin{array}{ccccccccccccccc} 1 & 2 & 3 & 4 & 5 & 6 & 7 & 8 & 9 & 10 & 11 & 12 & 13 & 14 & 15 \\
                       13& 2 &15&14&10& 6 &12& 3 & 4 &   1 &   7 &   9 &   5 & 11 &   8 \end{array} \right) $$
  $$      \tau = \left( \begin{array}{ccccccccccccccc} 1 & 2 & 3 & 4 & 5 & 6 & 7 & 8 & 9 & 10 & 11 & 12 & 13 & 14 & 15 \\
                          14& 9 &10& 2 &12 & 6 & 5 & 11 & 15 &   3 &   8 &   7 &   4 & 1 &   13 \end{array} \right). $$
Express these permutations in their disjoint cycle representation. Then 
compute $\sigma^2$, $\sigma \tau$, $\tau \sigma$, $\tau^2\sigma$, and $\sigma^{-1} \tau$ using their cycle representation. Determine whether
$\sigma$ and $\tau$ are even or odd permutations. 
\item Show that if $\sigma$ is a $k$-cycle then $|\sigma| = k$.
\item Prove that the order of an element in $S_n$ equals the least common multiple of the lengths of the cycles in its cycle decomposition.
\item Find all numbers $n$ such that $S_7$ contains an element of order $n$. 
\item Prove that $\sigma^2$ is an even permutation for every permutation $\sigma$. 
\item {\it Challenge}: Show that every $\sigma \in A_n$ is a product of $3$-cycles.

\end{enumerate}



\end{document}



