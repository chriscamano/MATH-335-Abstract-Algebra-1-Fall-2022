\documentclass[12pt]{article}
\usepackage[pdftex]{graphicx}
\usepackage{amsmath,amssymb,amsthm}
\usepackage{hyperref}
\pagestyle{empty}
\author{Chris Camano: ccamano@sfsu.edu}
\title{MATH 335  Lecture 2 }
\date

\topmargin -0.6in
\headsep 0.40in
\oddsidemargin 0.0in
\textheight 9.0in
\textwidth 6.5in
\vfuzz2pt
\hfuzz2pt

%%%%Short cuts and formatting%%%%%%%%%%
\newcommand{\q}{\quad}
\newcommand{\tab}{\\\\}
\renewcommand{\labelenumi}{\alph{enumi})}
\newcommand{\sect}[1]{\section*{#1}}

%%%%%%Vector Spaces%%%%%%%%%%%%%%%%%%%
\newcommand{\R}{\mathbb{R}}
\newcommand{\C}{\mathbb{C}}
\newcommand{\F}{\mathbb{F}}
\newcommand{\rtwo}{\mathbb{R}^2}
\newcommand{\mxn}{{m\times n}}

%%%%%%Sets and common phrases%%%%%%%%%
\newcommand{\Axb}{\textbf{Ax=b} }
\newcommand{\Axz}{\textbf{Ax=0} }
\newcommand{\dim}{\text{dim}}
\newcommand{\lc}{linear combination }
\newcommand{\let}{\text{Let }}
\newcommand{\tf}{\therefore}
%%%%%%%%%Analysis%%%%%%%%%%%%%%%%%%%%%
\newcommand{\arr}{\rightarrow}
\newcommand{\xlim}{\lim_{x\rightarrow \infty}}
\newcommand{\Z}{\mathbb{Z}}
\newcommand{\N}{\mathbb{N}}
\newcommand{\ep}{\varepsilon}
\newcommand{\i}{\text{ if }}
\newcommand{\and}{\text{ and }}
%%%%%% Theorem formatting%%%%%%%%%%%
\newtheorem{thm}{Theorem}[section]
\newtheorem{cor}[thm]{Corollary}
\newtheorem{lem}[thm]{Lemma}
\newtheorem{prop}[thm]{Proposition}
\theoremstyle{definition}
\newtheorem{defn}[thm]{Definition}
\theoremstyle{remark}
\newtheorem{rem}[thm]{Remark}
\numberwithin{equation}{section}
\everymath={\displaystyle}


\begin{document}
\maketitle
\prop Let $a,b \in \Z^+$ where a can be written as the following with P as a prime number:
\[
  a=\prod_{i=0}^kP_i^{\alpha_i}
\]
\[
  b=\prod_{i=0}^kP_i^{\beta_i}
\]
\[
  \alpha_i,\beta_i \geq 0\quad
\]
Then:
\[
  gcd(a,b)=\prod_{i=0}^kP_i^{\min\{\alpha_i,\beta_i\}}
\]
\begin{proof}\\
  Let d= $\prod_{i=0}^kP_i^{\min\{\alpha_i,\beta_i\}}$, We must show that d is a common divisor and that there does not exist a divisor greater than d. this is to say:
  \begin{itemize}
    \item $d|a$ and $d|b$
    \item if $e|a$ and $e|b$ then $e|d$
  \end{itemize}
  \begin{itemize}
    \item \begin{proof}\\
      $d|a$ and $d|b$\\
      To show this we mush show that $a=dc$ for some $c \in \Z$. \\
      \[
        a=\prod_{i=0}^kP_i^{\alpha_i}=\left(\prod_{i=0}^kP_i^{\min\{\alpha_i,\beta_i\}}\right)\left(\prod_{i=0}^kP_i^{\alpha_i-\min\{\alpha_i,\beta_i\}}\right)
      \]
      Where $$\alpha_i-\min\{\alpha_i,\beta_i\} \geq 0 \quad \forall \alpha_i$$
      By similar reasoning we can prove that d|b as well and the proof is complete.
      \end{proof}
      \item \begin{proof}
        if $e|a$ and $e|b$ then $e|d$\\
        Suppose $e\in \Z$ such that $e|a,e|b$. We now need to show $e|d$ Due to the fact that e divides both a and b this implies that the prime factorization of e is composed only of the prime factors of the two dividends, or in other words.
        \[
          e=\prod_{i=0}^k P_i^{\delta_i}
        \]
        Where $P_1,\ldots,P_k$ are primes only from the set of primes composed by the prime factorization of a and b and $\delta_i \leq \min (\alpha_i,\beta_i)$\\
        Then e|d since the exponents of the primes in d are equal to $\min (\alpha_i,\beta_i)$.
  \end{proof}
  \end{itemize}
  We have then proven the two conditions required to prove the greatest common divisor.
\end{proof}
Example:
\begin{align*}
  30&=2^1(3^1)(5^1)\\
  96&=2^5(3^1)(5^0)
\end{align*}
\defn Let $a,b \in \Z^+$ then an integer d is the least common multiple of a and b, d=lcm(a,b) $\iff$ two conditions are satisfied:
\begin{itemize}
  \item $a|d$ and $b|d$
  \item if $a|e$ and $b|e$ then $d|e$
\end{itemize}
\prop Let $a,b \in \Z^+$ with the prime factorizations:
\[
  a=\prod_{i=0}^kP_i^{\alpha_i}
\]
\[
  b=\prod_{i=0}^kP_i^{\beta_i}
\]
where $P_1,...,P_k$ are distinct primes and:
\[
  \alpha_i,\beta_i \geq 0\quad
\]
Then the least common multiple of a and b is equal to:
\[
  lcm(a,b)=\prod_{i=0}^kP_i^{\max\{\alpha_i,\beta_i\}}
\]
\end{document}
