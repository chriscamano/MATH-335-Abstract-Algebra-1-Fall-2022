\documentclass[11pt]{article}
\usepackage{amssymb,latexsym,amsmath,amsthm,graphicx, cite}
\author{Chris Camano: ccamano@sfsu.edu}
\title{MATH 310  Homework 7 }
\date

\usepackage[many]{tcolorbox}
\tcbset{breakable}
\usepackage{mathptmx}
\usepackage{multirow}
\usepackage{float}
\restylefloat{table}
\hoffset=0in
\voffset=-.3in
\oddsidemargin=0in
\evensidemargin=0in
\topmargin=0in
\textwidth=6.5in
\textheight=8.8in
\marginparwidth 0pt
\marginparsep 10pt
\headsep 10pt

\theoremstyle{definition}  % Heading is bold, text is roman
\newtheorem{theorem}{Theorem}
\newtheorem{corollary}{Corollary}
\newtheorem{defn}{Definition}
\newtheorem{example}{Example}
\newtheorem{proposition}{Proposition}



\newcommand{\Z}{\mathbb{Z}}
\newcommand{\N}{\mathbb{N}}
\newcommand{\Q}{\mathbb{Q}}
\newcommand{\R}{\mathbb{R}}
\newcommand{\C}{\mathbb{C}}
\newcommand{\lcm}{\mathrm{lcm}}
\setlength{\parskip}{0cm}
%\renewcommand{\thesection}{\Alph{section}}
\renewcommand{\thesubsection}{\arabic{subsection}}
\renewcommand{\thesubsubsection}{\arabic{subsection}.\arabic{subsubsection}}
\bibliographystyle{amsplain}

%\input{../header}
\newcommand{\bigline}{\\\noindent\makebox[\linewidth]{\rule{\textwidth}{0.4pt}}\\}

\newcommand{\block}[2]{\begin{tcolorbox}[title={#1}]{#2}\end{tcolorbox}}
\begin{document}
\maketitle
%%%%%%%%%%%%%%%%%%%%%%%%%%%%%%%%%%%%%%%%%%%%%%%%%%%%%%%%%%%%%%%%%%%%
\block{Question 1}{
Find all subgroups of $G=\langle a\rangle$ where $|a|=45$. Describe the containments between these subgroups.
}
\begin{proof}
  $ $\newline
   The unique subgroups of G are:
   \begin{align*}
     &<a^{45}>\\
     &<a^3>\\
     &<a^5>\\
     &<a^{15}>\\
     &<a^{9}>\\
     &<e>\\
   \end{align*}
   With the containment realtion:
   \[
     <e>\subseteq <a^{9}> \subseteq <a^3> \subseteq <a^{45}>
   \]
   \[
     <e>\subseteq <a^{15}> \subseteq <a^5> \subseteq <a^{45}>
   \]
   Note that :
  \[
    <a^{15}>\subseteq <a^3>
  \]
   As well.
\end{proof}
%%%%%%%%%%%%%%%%%%%%%%%%%%%%%%%%%%%%%%%%%%%%%%%%%%%%%%%%%%%%%%%%%%%%%%
\block{Question 2}{
Find all generators of $\mathbb{Z}_{48}$.
}
\begin{proof}
  $ $\newline
  To start there are a total of:
  \[
    \phi(48)=\phi(2^4)\phi(3)=8(2)=16
  \]
  Geneators since this is how many times we obtain a denominator of one when solving for the order of each element.
  The 16 relatively prime numbers to 48 are contained in the unit group of 48 therefore generators for $\Z_{48}$ are the elements of $U(48)$.
\end{proof}
%%%%%%%%%%%%%%%%%%%%%%%%%%%%%%%%%%%%%%%%%%%%%%%%%%%%%%%%%%%%%%%%%%%%%%
\block{Question 3}{
Let $\  left(G_1, \circ\right)$ and $\left(G_2, \bullet\right)$ be two groups with the respective group operations $\circ$ and $\bullet$. Show that the cartesian product $G_1 \times G_2$ is a group with the following operation:
$$
\left(a_1, b_1\right) \diamond\left(a_2, b_2\right):=\left(a_1 \circ a_2, b_1 \bullet b_2\right) .
$$
}
\begin{proof}
  $ $\newline
  To show that $G_1 \times G_2$ is a group we prove the following:
  \begin{enumerate}
    \item closure\\
    To demonstrate closure under the operation $\left(a_1, b_1\right) \diamond\left(a_2, b_2\right):=\left(a_1 \circ a_2, b_1 \bullet b_2\right) .$ Since $\left(G_1, \circ\right)$ and $\left(G_2, \bullet\right)$ are closed under their respective operators we have the fact that for any ordered pair produced by the cartersian product with our new element $a_1 \circ a_2 \in G_1$ and $ b_1 \bullet b_2\right \in $
    $G_2$ This implies that for all elements produced by our operator we obtain a new element of from the cartesian product of the two sets which is the desired meaning of closure in this context.
    \item  \textbf{Assoicativity}
    \begin{align*}
      &\text{Let } (a_1,b_1),(a_2,b_2),(a_3,b_3)\in G_1\times G_2\\\\
      &(a_1,b_1)((a_2,b_2)(a_3,b_3))=((a_1,b_1)(a_2,b_2))(a_3,b_3)\\\\
      &(a_1,b_1)(a_2\circ a_3,b_2\bullet b_3)=(a_1\circ a_2,b_1\bullet b_2)(a_3,b_3)\\\\
      &(a_1\circ a_2\circ a_3,b_1\bullet b_2\bullet b_3)=(a_1\circ a_2\circ a_3 ,b_1\bullet b_2\bullet b_3)\\\\
    \end{align*}
    \item \textbf{Identity}\\
    Consider the composition of the following :
    \[
      \left(a_1, b_1\right) \diamond\left(a_1^{-1}, b_2^{-1}):
    \]
    By definition of the operator we obtain:
    \[
      \left(a_1 \circ a_1^{-1},a_1^{-1} \bullet b_1^{-1}\right)=(e,e)
    \]
    \item \textbf{Inverse}\\
    Consider an element of $G_1 \times G_2$:
    \[
      (c,d)\in G_1 \times G_2
    \]
    By definition
    \[
      (c,d)=(a_1\circ a_2, b_1\bullet b_2)
    \]
    Since $a_1\circ a_2\in G_2$ so is c, and likewise since $b_1\bullet b_2 \in G_2$  so is d ,  Since $G_1$ and $G_2$ are groups there exists an inverse for each element in each group so we can construct an inverse:
    \[
      (c^{-1},d^{-1})\in G_1 \times G_2
    \]
    Proving the existence of an inverse:
    \[
        (c,d)\diamond (c^{-1},d^{-1})=(e,e)
    \]
  \end{enumerate}
  Thus $G_1 \times G_2$ is a group under $\left(a_1, b_1\right) \diamond\left(a_2, b_2\right):=\left(a_1 \circ a_2, b_1 \bullet b_2\right)$
\end{proof}
%%%%%%%%%%%%%%%%%%%%%%%%%%%%%%%%%%%%%%%%%%%%%%%%%%%%%%%%%%%%%%%%%%%%%%
\block{Question 4}{
 List the elements in the group $\Z_2 \times \Z_3$. Show that this group is cyclic.
}
\begin{proof}
  $ $\newline
  The elements of $\Z_2 \times \Z_3$ are:
  \[
    \Z_2 \times \Z_3=\{(0,0),(0,1),(0,2),(1,0),(1,1),(1,2)\}
  \]
  Assuming that the group operator over the ordered pairs is element wise addidtion with respect to the original modular base:
  the generator of this set is:
  \[
    <(1,1)>=\{(1,1),(0,2),(1,0),(0,1),(1,2),(0,0)
  \]
\end{proof}
%%%%%%%%%%%%%%%%%%%%%%%%%%%%%%%%%%%%%%%%%%%%%%%%%%%%%%%%%%%%%%%%%%%%%%
\block{Question 5}{
List the elements in the group $\Z_2 \times \Z_2$. Show that this group is not cyclic. Argue that by now we know two different abelian groups with four elements
}
\begin{proof}
  $ $\newline
  The elements of this group are as follows:

  \[
    \Z_2 \times \Z_2=\{(0,0),(0,1),(1,0),(1,1)\}
  \]
  To demonstrate that this is not a cycic group we will show that each subgroup is not a generator:
  \begin{align*}
    &<(0,0)>=\{(0,0)\}\\
    &<(0,1)>=\{(0,0),(1,1)\}\\
    &<(1,1)>=\{(0,0),(0,1)\}\\
    &<(1,0)>=\{(0,0),(1,0)\}\\
  \end{align*}
  This group itself is a non cyclic abelian group since the elements involved in the construction of the set come from abelian groups ( integers mod n ,+). This group has four elements so it is one of our order 4 abelian groups . \\

  One way to find another order 4 abelian group is to consider the cartesian product of  an abelian group with two elements and another abelian  group with with two elements:
  Consider = the unit group of 3 :
  \[
    U(3) \times U(3) =\{(1,1),(1,2),(2,1),(2,2))
  \]
  Which is an abelian group of order 4
\end{proof}
%%%%%%%%%%%%%%%%%%%%%%%%%%%%%%%%%%%%%%%%%%%%%%%%%%%%%%%%%%%%%%%%%%%%%%
\block{Question 6}{
Prove that neither $\Z_2 \times \Z$ nor $\Z \times \Z$ are cyclic groups.
}\begin{enumerate}
  \item \block{\textbf{Subproof 1: $\Z_2\times \Z$}}{
  \begin{proof}
    $ $\newline
    \[
      \Z_2=\{0,1\}
    \]
    \[
    \Z_2 \times \Z=\{(0,k),(1,k):k\in \Z\}
    \]
    By the definition of composition of groups under a cartesian product given in problem 3. \\
    Suppose that $\Z_2\times \Z$ is cyclic, then there exists (a,b)$\in \Z_2\times \Z$ such that :
    \[
      <(a,b)>=\Z_2\times \Z
    \]
    Consider the following:\\
    Since (a,b)$\in \Z_2\times \Z$ It is either of the form (1,k) or (0,k) where k is an integer.
    \\
    \textbf{Case 1: (a,b) is of the form (1,k):}\\
    If (a,b) is of the form (1,k) then the set generated by (a,b) does not contain (0,k) $\forall k \in \Z$ which is a contradiction \\
    \textbf{Case 2: (a,b) is of the form (0,k):}\\
    If (a,b) is of the form (0,k) then the set generated by (a,b) does not contain (1,k) $\forall k \in \Z$ which is a contradiction \\
    Thus in either case we arrive to a contradiction meaning there does not exist a generator for$\Z_2\times \Z$ implying it is not cyclic.
  \end{proof}}
  \item \block{\textbf{Subproof 1: $\Z\times \Z$}}{
  \newcommand{\ztz}{}
  \begin{proof}
    $ $\newline
    To demonstrate that $\Z\times \Z$ is not a cyclic group we will proceed with a proof by cotradiction levergaining the fact that if a group is cyclic then there exists a generator: \\
    Suppose that $\Z\times \Z$ is cyclic, then by definition there exists an ordered pair in $\Z\times \Z$ (a,b) such that :
    \[
      <(a,b)>=\Z \times \Z
    \]
    If $a=0$ then $(1,0)$ is not in this set, which leads to a contradiction since (a,b) is a supposedly a generator. So we have $a \neq 0$.\\
    If $b=0$ then $(0,1)$ is not in this set, which leads to a contradiction since (a,b) is a supposedly a generator. So we have $b \neq 0$.\\
    Consider the element $$(a,-b) \in \mathbb{Z} \times \mathbb{Z}$$There is an integer $k \in \mathbb{Z}$ with $(k a, k b)=(a,-b)$, and since $a, b \neq 0$ this gives $k=1$ and $k=-1$, which is a contradiction.\\\\
    Together we have shown that element (0,1) cannot be generated with this construction
  \end{proof}}
\end{enumerate}


%%%%%%%%%%%%%%%%%%%%%%%%%%%%%%%%%%%%%%%%%%%%%%%%%%%%%%%%%%%%%%%%%%%%%%
\block{Question 7}{
Let $G$ be a group and let $C_1 = \langle a \rangle$ and $C_2 = \langle b \rangle$ be two cyclic subgroups with orders $n$ and $m$, respectively.
 Prove that if $\gcd(n,m) = 1$ then $C_1 \cap C_2 = \{e\}$.
 }
 \begin{proof}
   $ $\newline
  If $|<a>|=n$ this implies $a^n=e$ likewise   If $|<b>|=m$ this implies $a^m=e$. This knowledge together allows us to build representations of the elements of the group:
  \[
    C_1=\{e, a, a^1,a^2,\cdots, a^{n-1}\}
  \]
  \[
    C_2=\{e, b, b^1,b^2,\cdots, b^{m-1}\}
  \]
  Suppose there exists another element in the intersection of $C_1$ and $C_2$, 
   Since we have the property that:

   \[
     |a^k|=\frac{n}{\gcd(n,k)}
   \]
   This means that:
   \[
     |a^k|\gcd(n,k)=n\rightarrow |a^k||n
   \]
  So this means that since $x\in C_1$ that $$|x||n$$ as $x=a^l,l\in[1,n-1]$\\
  Likewise since $x\in C_2$ that $$|x||m$$ as $x=b^i,i\in[1,m-1]$\\
  But gcd(n,m)=1, meaning that the only time this could be true is when x=1=e
 \end{proof}
%%%%%%%%%%%%%%%%%%%%%%%%%%%%%%%%%%%%%%%%%%%%%%%%%%%%%%%%%%%%%%%%%%%%%%
\end{document}
